%!TEX root = ../../../memoria.tex

\subsection{\workflowsCPT}

%TODO
% Introduccion de workflows

Existen una serie de flujos que se pueden experimentar al visitar un sitio \ecommerceCOM, entre los cuales podemos destacar:

	\begin{itemize}
		\item
			Shopping Experience.
		\item
			Proceso de una orden
		\item
			\shipping.
		\item
			Sistema de pago.
		\item
			Servicio al Cliente
		\item
			Retorno de artículos.
		\item
			Entre otros. 
	\end{itemize}


Cada uno de los pasos dentro de un \workflowsCPT puede ser considerado como un estado el cual cambiará dependiendo de los eventos que esten involucrados. Por lo tanto cada uno de estos \workflowsCPT puede ser modelado utilizando una máquina de estados finitos.
Para el caso particular del \frameworkPC para \ecommerceCOM, se han modelado 2 \workflowsCPT que se implementan utilizando la librería \javaScriptNAME \finiteStateMachine.

\begin{figure}[H]
	\centering
	\includegraphics[width=1.1\textwidth]{figuras/cart_state_machine.jpg}

	\caption{Máquina de estado de carro de compras.}
	\label{figure:cart_state_machine}
\end{figure}


\begin{figure}[H]
	\centering
	\includegraphics[width=0.8\textwidth]{figuras/order_state_machine.jpg}

	\caption{Máquina de estado del estado de una orden.}
	\label{figure:order_state_machine}
\end{figure}

Notar que en estricto rigor, estos diagramas no corresponden a una máquina de estados finitos, sin embargo, es una excelente estrategia abordar estos procesos como máquinas de estados.