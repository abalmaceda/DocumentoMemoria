
\chapter{Avances}

\section{Arquitectura}

\packagesAS es lo ultimo en \codeSeparationQA, \modularityAS, y \reusabilityQA. Al estructurar el código de cada \featureCPT en un \packagesAS separado, el código de una \featureCPT no tendrá acceso al código de otra \featureCPT excepto utilizando \exportCPT, haciendo cada dependencia explícita. Además esto permite la manera mas sencilla para realizar \testingCPT independientes a cada \featureCPT. Incluso es posible \publishINT los \packagesAS y utilizarlos en otros proyectos.

Es esta forma de estructurar el código la parte principal del desarrollo del \frameworkPC \ecommerce. Cada vez que se requiera desarrollar una nueva solución específica utilizando el \frameworkPC, simplemente se incluirán todos aquellos \modulesAS necesarios para el desarrollo de la aplicación, y con un poco de desarrollo, obtener las \featuresCPT específicas buscadas.

Como este \frameworkPC será completamente \openSourcePC, la comunidad podrá crear nuevos \modulesAS con el fin de crear aquellas \featuresCPT que serán requeridos en el futuro.



\subsection{\packagesAS}

A continuación se describe las funcionalidades de cada uno de los \packagesAS que actualmente existen. La estructura se encuentra en \refFigura{cap:avances:current_architecture}.

\tikzset{
  basic/.style  = {draw, text width=2cm, drop shadow, font=\sffamily, rectangle},
  root/.style   = {basic, rounded corners=2pt, thin, align=center,
                   fill=green!30},
  level 2/.style = {basic, rounded corners=6pt, thin,align=center, fill=green!60,
                   text width=8em},
  level 3/.style = {basic, thin, align=left, fill=pink!60, text width=6.5em}
}
\begin{figure}[H]
	\centering
	
	
	\begin{tikzpicture}[
	  level 1/.style={sibling distance=40mm},
	  edge from parent/.style={->,draw},
	  >=latex]

	% root of the the initial tree, level 1
	\node[root] {\rtcom}
	% The first level, as children of the initial tree
	  child {node[level 2] (c1) {\rtcomCorePCKG}}
	  child {node[level 2] (c2) {\helloworldPCKG}}
	  child {node[level 2] (c3) {\rtcomSearchPCKG}}
	  %child {node[level 2] (c4) {rtcom-shipping}}
	 % child {node[level 2] (c5) {rtcom-paypal}}
	  child {node[level 2] (c6) {\rtcomGoogleanalPCKG}};

	\end{tikzpicture}
	\caption{Arquitectura actual de la solución.}
	\label{cap:avances:current_architecture}
\end{figure}
	\begin{itemize}
		\item
			\textbf{\rtcom} es un \frameworkPC \ecommerce desarrollado con \meteorNAME, \nodejsNAME, \mongodbNAME, \coffeescript con un enfoque \reactive y \realTimeINT.
		\item
			\textbf{\rtcomCorePCKG} corresponde al \coreAS de la solución. Contiene todas aquellas \featuresCPT básicas y \templatesAS a una plataforma \ecommerce tales como manejo de \itemsCOM, \sessionsINT, etc. De momento he trabajado solo con esas dos \featuresCPT.
		\item
			\textbf{\rtcomGoogleanalPCKG} es el \moduleAS para \googleanalytics. Aunque no se encuentra completamente integrado, si genera datos.
		\item
			\textbf{\helloworldPCKG} es un \moduleAS para prueba de conceptos.
	\end{itemize}

\section{\dataModelAS}

Existen \dataModelsAS que se encuentran en una fase avanzada de desarrollo. Recordar que la \dataBaseDB (\mongodbNAME), no tiene \schemasDB estáticos. Esto ha sido particularmente valioso en el proceso de desarrollo de la solución, dado que variados cambios han surgido en el proceso. Estos \dataModelsAS pueden ser revisados en \refApendice{ap:avance_data_model}.

