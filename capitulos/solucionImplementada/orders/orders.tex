%!TEX root = ../../../memoria.tex
\section{\ordersEF}\label{chapter:solucionimplementada:section:orders}
%  FIXME : Nueva información agregada despues de revision de mi papa
Muestra información relevante relacionada con cada una de las compras realizadas en el sitio por el cliente autenticado. Esta interfaz es equivalente a la que utiliza el administrador desde el \dashboardEF con la gran diferencia que acá el usuario solo ve las \ordersEF que el ha generado en el sistema, en cuanto el administrador, ve todas las \ordersEF generadas en el sistema. (ver \refFigura{figure:dashboard:orders:grid}). Este escenario no es el correcta, dado que las necesidades de estos usuarios son distintas, por lo tanto ambas interfaces no pueden ser exactametne igual. Sin embargo, esta plataforma ha sido desarrollada para el contexto de una Memoria, por lo que no es posible la implementación total de cada Interfaz.

 % Los detalles de esta interfaz fueron inspirados en la página de \ordersEF del \websiteINT \dealextremeNAME (ver \refFigura{figure:apendice:orders:example:dx_list_orders}).  


% \begin{figure}[H]
% 	\centering
% 	\includegraphics[width=0.1\textwidth]{figuras/orders/client/list_orders.jpg}
% 	\caption{Lista de las \ordersEF que tiene un cliente.}
% 	\label{figure:orders:client:list_orders}
% \end{figure}

% Es importante recordar que las \ordersEF desplegadas corresponden a las del usuario autenticado.