%!TEX root = ../../../memoria.tex

%you can provide your own UI for these features or you can even use a premade boilerplate UI to get up and running quickly.

%eteor Accounts is a modular system, and anyone can write a package that provides a new login method. Some, but not all of these packages are officially maintained by the Meteor Project. For example, accounts-password has a complete password-based login system, including password recovery. Packages such as accounts-google, accounts-facebook, and accounts-twitter provide the ability to log in with common third-party authentication services.

%The account database is stored in a users database collection that is automatically created on the server. Currently this can only be stored in MongoDB, but other databases will likely be supported in the future, depending on demand. The schema of this database is shown in the main Meteor docs.

%Passwords are safely encoded with bcrypt according to industry best practices.

\section{Localización}

Ingles no solo ostenta el primer lugar como el idioma más utilizado. \rankingCPT que considera además del porcentage de población, la distribución geográfica. Si no que además es el idioma más utilizado en los \websitesINT alcanzando la cifra de 59,4\%, seguido tímidamente por Rusia con un 5.9\% \cite{online_world_wide_languages}. Mirando estas cifras se podría concluir que la inclusión de múltiples idiomas en un \siteINT \ecommerceCOM	no reprensentaría ingresos importantes.
El potencial económico \online es de \$45 trillones, de acuerdo a un estudio realizado por \commonSenseAdvisoryNAME \cite{online_world_global_oportunity_multi_languages}.  Sin una solida estrategia de localización que considere \websitesINT \ecommerceCOM con \textit{multi-idiomas}, será  un desafio tener al alcanze de la mano ese \revenueCOM. De hecho, el mismo estudio vislumbró que si solo se tiene una versión en ingles del \siteINT, se estará limitado solo a una tercera parte del total.

%Grow revenue with website localization
%The economic potential of online communication is $45 trillion, according to a recent study by Common Sense Advisory. How exciting! The sky is the limit for your global business.
%Or is it?
%Without a solid localization strategy that incorporates an e-commerce website in multiple languages, it will be a challenge to get within arm’s reach of this revenue. In fact, the same study found that if you have just an English version of your website, you’re limited to only one third of the pot.

¿Cuántos serán los idiomas necesarios para permanecer competitivo en el mundo \online. Los investigadores dicen que un mínimo de 14. Las marcar mundiales que aspiran a un 95\% de las billeteras \online necesitan de 20 idiomas. Si bien es cierto, no es siempre posible mantener contenido \online en 20 idiomas, es difícil negar los beneficios de la localización para regiones específicas al rededor del mundo \cite{online_world_global_oportunity_multi_languages}. Estan son las razones de por que se considera muy importante la localización en el \frameworkPC y de por que se aborda esta caracteristica desde los inicios.

Actualmente la aplicación cuenta con un sistema robusto para la localización, permitiendo nuevos idiomas simplemente agregando un archivo \jsonNAME. En la \refFigura{figure:features:languages_available}
%How many languages does it take for global businesses to stay competitive online? The research says a minimum of 14. Global brands wanting to appeal to 95 percent of the world’s online wallet need 20 languages. While it isn’t always feasible for a business to translate online content into 20 languages, it’s hard to deny the benefits of website localization for targeted regions around the world.
%Many businesses recognize this and plan to add even more languages to their localization strategy in the future. One example is European-based clothing seller ASOS. They launched a website for the Chinese and Russian markets in 2013 to expand their global footprint. They already had a presence in the United Kingdom, United States, France, Germany and Australia. They saw international sales overall increase by 39 percent with their past website localization initiatives—so they knew that additional language sites would help grow revenue.
%If you’re ready to get a bigger piece of the global e-commerce pie, like ASOS, consider upping the number of languages on your website. Check out the article, Global e-commerce: Are these 5 items in your localization shopping cart?, for tips on how to approach e-commerce website localization.


\begin{figure}[H]
	\centering
	\includegraphics[width=0.3\textwidth]{figuras/languages_available.png}

	\caption{Selección de idioma para el \websitesINT.}
	\label{figure:features:languages_available}
\end{figure}




\section{\themesCPT}

Como se explicó en la subsección \ref{chapter:section:subsection:package_core_theme}, la aplicación permite cambiar el \themeCPT utilizando el \packageAS correspondiente.
Una de las partes más importantes del \frameworkPC \ecommerceCOM corresponde justamente a la customización de la aplicación, dado que esto lograr generar identidad.
Posteriormente el sistema permitirá además agregar \templatesAS que permita una personalizaciónn del \websiteINT aun más detallada.



\section{La información persiste}

Aunque me cambie de vista , si regreso, la información deberia seguir viendose. Por ejemplo si lleno un formulario y me voy, al regresar, este formulario estara igual


%TODO: ver si es necesario agregar esto
% caracteristicas actuales en formularios
%%!TEX root = ../../../memoria.tex
\section{Interfaz de Usuario}

Este proyecto no tiene entre sus objetivos la usabilidad de sus interfaces, razón por la cual no es necesario presentar resultados de experiencia de usuario en esta materia, sin embargo, una buena interfaz de usuario se considera buena si tiene un alto \conversionRateCOM y es \textit{fácil de usar} \cite{online_goodgui_org}


% caracteristicas actuales en formularios
%!TEX root = ../../../../../memoria.tex
\section{Formularios}

Los formularios son parte relevante de la aplicación. Por lo tanto se puse especial énfasis en 

\subsection{Pocos campos en los formularios}
Cada uno de los formularios que existen en la aplicación fueron creados con este principio. Los seres humanos son intrínsecamente resistentes a labores de tareas intensivas y el mismo principio se aplica para llenar completamente un formulario. Por cada campo agregado al formulario, se corre el riesgo de perder un cliente\cite{online_goodgui_org}.
%TODO : Agregar referencia a figura de creación de usuario
Un caso claro de esto, corresponde a la interfaz de creación de un usuario ( \refFigura{figure:account:create_account:form}), la cual fue diseñada solo con dos campos.


\subsection{Errores}
Los errores ocurriran sin importar las medidas que se tomen. Lo importante es evitar al máximo que estos sucedan. Y cuando esto pase, guiar adecuadamente al usuario para que pueda resolver los problemas sin mayores inconvenientes. Para estos casos se debe proceder de la manera siguiente:
\begin{itemize}
	\item Comunicar claramente que es aquello que esta sucediendo.
	\item Describir como el usuario puede resolverlo.
	\item Mantener tanto \inputBrowserINT del usuario como sea posible \cite{online_google_ui_pattern_error}.
\end{itemize} 


\begin{figure}[H]
	\centering
	\includegraphics[width=0.5\textwidth]{figuras/formularios/form_address_book.png}

	\caption{Formulario de creación de una nueva dirección de envio.}
	\label{figure:form:form_address_book}
\end{figure}
















