%!TEX root = ../../../memoria.tex

\tikzset{
  basic/.style  = {draw, text width=2cm, drop shadow, font=\sffamily, rectangle},
  root/.style   = {basic, rounded corners=2pt, thin, align=center,
                   fill=green!30},
  level 2/.style = {basic, rounded corners=6pt, thin,align=center, fill=green!60,
                   text width=8em},
  level 3/.style = {basic, thin, align=left, fill=pink!60, text width=6.5em}
}
\begin{figure}[H]
	\centering
	\begin{tikzpicture}[
	  level 1/.style={sibling distance=40mm},
	  edge from parent/.style={->,draw},
	  >=latex]

	% root of the the initial tree, level 1
	\node[root] {\eframeworkAF}
	% The first level, as children of the initial tree
		child {node[level 2] (c1) {\eframeworkCorePCKG}}
		%child {node[level 2] (c2) {\eframeworkHelloWorldPCKG}}
		child {node[level 2] (c3) {\eframeworkAccountsPCKG}}
		%child {node[level 2] (c4) {rtcom-shipping}}
		child {node[level 2] (c5) {\eframeworkPaypalPCKG}};
		%child {node[level 2] (c6) {\eframeworkGoogleanalPCKG}};
	\end{tikzpicture}
	\caption{\architectureCPT actual de la solución.}
	\label{cap:avances:current_architecture}
\end{figure}


%!TEX root = ../../memoria.tex
\begin{figure}[h!]
	\centering
	\scaleCommandTreeFolder
		\tikzstyle{every node}=[draw=black,thick,anchor=west,inner sep=2pt,minimum size=1pt]
		\tikzstyle{selected}=[draw=cyan,fill=cyan!30]
		\begin{tikzpicture}[scale=(\scaleValueTreeFolder),
			grow via three points={one child at (0.5,-0.7) and
			two children at (0.5,-0.7) and (0.5,-1.4)},   
			edge from parent path={(\tikzparentnode.south) |- (\tikzchildnode.west)}]
			\node {\eframeworkAF}
				child { node {\eframeworkCorePCKG}}
				child { node {\eframeworkAccountsPCKG}}
				child { node {server/}}
				child { node {private/}}
				child { node {public/}}
				child { node {common/}}
				child { node {lib/}}
				child { node {docs/}}
				child { node {settings/}}
				child { node {packages/}}
				child { node [draw=none] {package.js} }
				child { node [draw=none] {README.md} }
				child { node [draw=none] {LICENSE.md} }
		    ;
		\end{tikzpicture}
		\tikzstyle{every node}=[] % resets borders of tables
		\tikzstyle{selected}=[] % resets selected
	\caption{Estructura global para una aplicación en \meteorNAME.}
	\label{esquema:gloal_structure_meteor_app}
\end{figure}