%!TEX root = ../../../memoria.tex
\section{Arquitectura del \frameworkPC}\label{cap:arquitectura:section:arquitectura_framework}

	\subsection{\packagesAS}

		A continuación se describen las funcionalidades de cada uno de los \packagesAS que actualmente existen. La estructura se encuentra en \refFigura{esquema:current:packages}.

		% Sección que habla sobre los workflows en el framwork
		%!TEX root = ../../../../memoria.tex
\begin{figure}[h!]
	\centering
	\scaleCommandTreeFolder
		\tikzstyle{every node}=[draw=black,thick,anchor=west,inner sep=2pt,minimum size=1pt]
		\tikzstyle{selected}=[draw=cyan,fill=cyan!30]
		\begin{tikzpicture}[scale=(1),
			grow via three points={one child at (0.5,-0.7) and
			two children at (0.5,-0.7) and (0.5,-1.4)},   
			edge from parent path={(\tikzparentnode.south) |- (\tikzchildnode.west)}]
			\node {\eframeworkAF}
				child { node {\eframeworkCorePCKG}}
				child { node {\eframeworkAccountsPCKG}}
				child { node {\eframeworkCoreThemePCKG}}
				% child { node {\eframeworkBootstrapThemePCKG}}
				child { node {\eframeworkShippingPCKG}}
				child { node {\eframeworkPaypalPCKG}}
		    ;
		\end{tikzpicture}
		\tikzstyle{every node}=[] % resets borders of tables
		\tikzstyle{selected}=[] % resets selected
	\caption{Estructura de los \packagesAS del \eframeworkAF.}
	\label{esquema:current:packages}
\end{figure}

		% \begin{itemize}
		% 	\item
		% 		\textbf{\eframeworkAF} es un \frameworkPC \ecommerceCOM desarrollado con \meteorNAME, \nodejsNAME, \mongodbNAME, con un enfoque \reactive y \realTimeINT.
		% 	\item
		% 		\textbf{\eframeworkCorePCKG} 
		% 	\item
		% 		\textbf{\eframeworkAccountsPCKG} es el \moduleAS para el manejo de cuentas.
		% 	\item
		% 		\textbf{\eframeworkCoreThemePCKG}
		% 	\item
		% 		\textbf{\eframeworkBootstrapThemePCKG}
		% 	\item
		% 		\textbf{\eframeworkShippingPCKG} es el \moduleAS encargado de las características de \ShippingCOM.
		% 	\item
		% 		\textbf{\eframeworkPaypalPCKG} es el \moduleAS del método de pago de \paypalNAME.
			
		% 	% \item
		% 	% 	\textbf{\eframeworkHelloWorldPCKG} es un \moduleAS para prueba de conceptos.
		% \end{itemize}


		La arquitectura descrita en la \refSection{cap:arquitectura:section:generic_arquitectura} fue diseñada para facilitar las implementaciones de cada uno de los \packagesAS que se desarrollen en los proyectos de \meteorNAME. Cada uno de los \packagesAS que se aprecian en la \refFigura{esquema:current:packages} hacen uso de dicha arquitectura.

		\subsubsection{\eframeworkCorePCKG}

			%Como su nombre lo indica, este \packagesAS corresponde al \coreAS del \frameworkPC, que contiene todas aquellas características intrínsecas que dan vida a un sitio \ecommerceCOM.
			Corresponde al \coreAS de la solución. Contiene varias características básicas de una plataforma \ecommerceCOM, además de  algunas funcionalidades genericas como:

			\begin{itemize}
				\item
					Manejo de \itemsCOM.
				\item
					Manejo de \sessionsINT. 
				\item
					Proceso de \checkoutCOM.
				\item
					Manejo de órdenes de Compra.
				\item
					Manejo del carro de Compra.	
				\item
					Manejo de las funcionalidades (\dashboardEF).
				\item
					Localización.
			\end{itemize}

			Se hablará en detalle de cada una de estas componentes.

		\subsubsection{\eframeworkAccountsPCKG}

			Este \packagesAS  esta enfocado en el manejo de cuentas de los usuarios. Permite entre otras cosas:

			\begin{itemize}
				\item
					Crear cuentas y editar cuentas.
				\item
					Asignar permisos a los usuarios del sistema.
				\item
					Recuperación de contraseña.
				\item
					etc.
			\end{itemize}

		\subsubsection{\eframeworkShippingPCKG}
			Este \packagesAS se encarga de las características relacionadas con \ShippingCOM, en otras palabras, permite la creación, edición y eliminación de opciones de \ShippingCOM.

		\subsubsection{\eframeworkPaypalPCKG}
			Este \packagesAS permite al sistema utilizar una de las varias opciones de pago que permite \paypalNAME. En la \refSection{chapter:solucionImplementada:dashboard:payment:subsection:paypal_pro} se hablará en detalle sobre el método de compra disponible.

		% \subsubsection{\eframeworkBootstrapThemePCKG}
		% 	Este \packagesAS esta encargado de la customización de las interfaces utilizando \bootstrap.

		\subsubsection{\eframeworkCoreThemePCKG}\label{chapter:section:subsection:package_core_theme}

		Este \packagesAS es la base del \bootstrap \themeCPT del \frameworkPC de \ecommerceCOM. Este contiene todos los archivos \lessNAME utilizados para el proceso que genera los archivos \lessNAME personalizados para el \frameworkPC. Este \packagesAS puede ser copiado para crear \themeCPT adicionales para el \frameworkPC \ecommerceCOM.

		La implementación de este \packagesAS está soportada sobre \bootstrapPackage permitiendo sencillamente hacer todas aquellas configuraciones que se desean. \bootstrapPackage permite, entre otras cosas, indicar exactamente cuáles son las componentes de \bootstrapNAME que se desean utilizar, así como personalizar e importar el \themeCPT.
			
			%Posteriormente el sistema permitirá además agregar \templatesAS que permitan una personalizaciónn del \websiteINT aun más detallada.
			

		% Sección que habla sobre los workflows en el framwork
		%!TEX root = ../../../memoria.tex

\subsection{\workflowsCPT}

% Introduccion de workflows

Existen una serie de flujos que se pueden experimentar al visitar un sitio \ecommerceCOM, entre los cuales podemos destacar:

	\begin{enumerate}
		\item
			Experiencia de \shoppingCOM.
		\item
			Proceso de una orden
		\item
			\shipping.
		\item
			Sistema de pago.
		\item
			Servicio al Cliente
		\item
			Retorno de artículos.
		\item
			Entre otros. 
	\end{enumerate}


Cada uno de los pasos dentro de un \workflowsCPT puede ser considerado como un estado, el cual cambiará dependiendo de los eventos que estén involucrados. Por lo tanto cada uno de estos \workflowsCPT puede ser modelado utilizando una máquina de estados finitos.
Para el caso particular del \frameworkPC para \ecommerceCOM, se han modelado 2 \workflowsCPT que se implementan utilizando la librería \javaScriptNAME \finiteStateMachine.

\begin{figure}[H]
	\centering
	\includegraphics[width=1.1\textwidth]{figuras/cart_state_machine.jpg}

	\caption{Máquina de estados de carro de compras.}
	\label{figure:cart_state_machine}
\end{figure}


\begin{figure}[H]
	\centering
	\includegraphics[width=0.8\textwidth]{figuras/order_state_machine.jpg}

	\caption{Máquina de estados del estado de una orden.}
	\label{figure:order_state_machine}
\end{figure}

Notar que en estricto rigor estos diagramas no corresponden a una máquina de estados finitos, sin embargo, es una excelente estrategia abordar estos procesos como máquinas de estados.