%!TEX root = ../../../memoria.tex
\section{\dashboardEF}
	Corresponde al administrador de características de la aplicación. Básicamente muestra todas las funcionalidades customizables que existen en la plataforma.
	%Cada package agregado el proyecto contiene el archivo \packageDescriptionFILE, el cual proporciona al \dashboardEF la información necesaria para agregar una o más componentes a la \uiSiglaAS con información descriptiva y con el link hacia la pantalla de configuración.
	La información que se muestra de cada componente corresponde a:

	\begin{itemize}
		\item Nombre de la componente a agregar.
		%\item Descripción breve de la componente.
		\item Un icono.
	\end{itemize} 

	En la \refFigura{figure:dashboard:main_menu} se observa el \dashboardEF 5 elementos configurables de la aplicación. 

	\begin{figure}[!h]
		\centering
		\includegraphics[width=0.6\textwidth]{figuras/dashboard/main_menu.png}
		\caption{\dashboardEF con elementos configurables de la aplicación.}
		\label{figure:dashboard:main_menu}
	\end{figure}

	Existen ciertos \packagesAS que tienen la opción de ser deshabilitados. Esta opción es especialmente útil para agregar o quitar ciertas funcionalidades al \websiteINT. A modo de ejemplo, en la \refFigura{figure:dashboard:paypal_disabled} se observa que el \packageAS de \paypalNAME está deshabilitado, en contraste con el \packageAS de \ShippingCOM. En el caso de que existieran más opciones de pago como \googleWalletNAME y \amazonPayments, se podrían ocultar los métodos sencillamente utilizando este método.

	\begin{figure}[!h]
		\centering
		\includegraphics[width=0.4\textwidth]{figuras/dashboard/paypal_disabled.png}
		\caption{\dashboardEF con elementos configurables de la aplicación.}
		\label{figure:dashboard:paypal_disabled}
	\end{figure}

	% Dashboar Core
	\input{capitulos/solucionImplementada/dashboard/core/core.tex}

	% Dashboar Orders
	%!TEX root = ../../../../memoria.tex
\subsection{\ordersEF}

	Por definición, una orden corresponde a una solicitud confirmada, en esta caso particular, desde un cliente para comprar un \itemCOM o servicio bajo unos términos y condiciones específicas. Es importante mencionar que el flujo de compra no ha terminado cuando el cliente a realizado el pago. Es en este escenario en donde surge el concepto de \orderFulfillmentCOM.
	Esta sección permite al encargado administrar las ordenes generadas por los clientes para posteriormente gestionarlas, conteniendo por lo tanto, toda la información necesaria para llevar a cabo dicho proceso.
	Dada la complejidad de operaciones que puede alcanzar \orderFulfillmentCOM, en la aplicación, está sección se ha simplificado a lo más fundamental.

	Al selección las Ordenes desde el panel de \dashboardEF, lo primero que se vera es la lista de todas las Ordenes ingresadas por los clientes que se encuentran actualmente en el sistema. En la \refFigura{figure:dashboard:orders:grid} se ve la pantalla de Ordenes.


	\begin{figure}[h!]
		\centering
		\includegraphics[width=0.6\textwidth]{figuras/orders/grid/main.png}
		\caption{Vista general de todas las órdenes ingresadas por clientes.}
		\label{figure:dashboard:orders:grid}
	\end{figure}

	La Componente visual de una \orderEF se encuentra áltamente influenciada por la interfaz de detalle de una \orderEF del sitio \dealextremeNAME. Con el fín de simplificar espació, se tomó el resumen de la compra y se movió desde el costado inferior derecho, al superior derecho. La \refFigura{figure:dashboard:orders:single_order} muestra la interfáz de una orden.

	\begin{figure}[h!]
		\centering
		\includegraphics[width=0.7\textwidth]{figuras/orders/grid/order.png}
		\caption{Orden ingresada por un cliente.}
		\label{figure:dashboard:orders:single_order}
	\end{figure}

% 	Cada una de estas órdenes está disponible para realizar acciones sobre ellas. Estas acciones radican principalmete en cambiar el estado actual de la orden. El detalle de la orden se ve en la \refFigura{figure:dashboard:orders:orderInfo}.
% %TODO agregar mas información sobre la orden.
% 	\begin{figure}[h!]
% 		\centering
% 		\includegraphics[width=0.6\textwidth]{figuras/dashboard/orders/orderInfo.png}
% 		\caption{Detalle de orden ingresada por un cliente.}
% 		\label{figure:dashboard:orders:orderInfo}
% 	\end{figure}

	% Dashboar accounts
	%!TEX root = ../../../../memoria.tex
\subsection{\accountsEF}

	Permite agregar o quitar permisos a los usuarios. El sistema actualmente permite crear usuarios y modificar sus permisos permitiendo o eliminando las componentes relacionadas a sus permisos.

	\begin{figure}[H]
		\centering
		\includegraphics[width=0.6\textwidth]{figuras/dashboard/account/users.png}
		\caption{Interfaz con los usuarios del sistema.}
		\label{figure:dashboard:account:users}
	\end{figure}


	\begin{figure}[H]
		\centering
		\includegraphics[width=0.6\textwidth]{figuras/dashboard/account/permisos.png}
		\caption{Interfaz para administrar los permisos del sistema.}
		\label{figure:dashboard:account:permisos}
	\end{figure}

\begin{table}[H]
    \centering
	\begin{tabular}{ |l|c||l| }
		\hline Campo & Requerido & Restricción \\ \hline
		\multirow{1}{*}{\textit{Core}} 			&  \checkmark 	& Boolean \\ \hline
		\multirow{1}{*}{\textit{Dashboard}} 	&  \checkmark	& Boolean \\ \hline
		\multirow{1}{*}{\textit{Orders}} 		&  \checkmark	& Boolean \\ \hline
		\multirow{1}{*}{\textit{Add Product}} 	&  \checkmark	& Boolean \\ \hline
		\multirow{1}{*}{\textit{Accounts}} 		&  \checkmark	& Boolean \\ \hline
		\multirow{1}{*}{\textit{Profile}} 		&  \checkmark	& Boolean \\ \hline
		\multirow{1}{*}{\textit{Shipping}} 		&  \checkmark	& Boolean \\ \hline
	\end{tabular}
 	\caption{Resumen restrincciones formulario para los permisos.}
    \label{tab:dashboard:account:form:restrictions:account}
\end{table}

	% Dashboar shipping
	%!TEX root = ../../../../memoria.tex
\subsection{\shippingEF}\label{cap:solucionImplementada:section:dashboard:subsection:shipping}

\shippingEF es parte del proceso realmente grande y complejo perteneciente al \workflowCPT \orderFulfillmentCOM. 

Como primer paso, hay que agregar una dirección de origen, la cual es agreagada en \nameref{capitulo:solucionImplementada:dashboard:subsubsection:addressPanel}.

Segundo, se deben agregar \shippingZonesCOM los cuales tendran su correspondiente \shippingRatesCOM. Existen servicios que permiten calcular en tiempo real el \shippingRatesCOM, evitando así que la tienda asuma un costo extra por un error en la estimación de la tarifa de envío. El sitio además debería ser capaz de determinar, si la dirección de destino se encuentra dentro de alguna de las \shippingZonesCOM definidas.

% TODO falta agregar descripcion del proceso
%https://docs.shopify.com/manual/shipping/initial-shipping-setup#add-a-shipping-zone

Por temás de tiempo, se decidío simplificar la solución, permitiendo agregar diferentes opciones de \shippingEF que eventualmente representarían diferentes criterios de envío.

%TODO Agregar mas informaciñon 

\begin{figure}[H]
	\centering
	\includegraphics[width=0.6\textwidth]{figuras/dashboard/shipping/shipping_options.png}
	\caption{Tabla con todas las opciones de \shippingEF disponibles.}
	\label{figure:dashboard:shipping:shipping_options}
\end{figure}

\begin{table}[H]
    \centering
	\begin{tabular}{ |l|c||l| }
		\hline Campo & Requerido & Restricción \\ \hline
		\multirow{1}{*}{\textit{Method Name}} 	&  \checkmark 	& \\ \hline
		\multirow{1}{*}{\textit{Public Label}} 	&  \checkmark	& \\ \hline
		\multirow{1}{*}{\textit{Group}} 		&  \checkmark	& \\ \hline
		\multirow{1}{*}{\textit{Cost}} 			&  				& \\ \hline
		\multirow{1}{*}{\textit{Handling}} 		&  				& \\ \hline
		\multirow{1}{*}{\textit{Rate}} 			&  \checkmark	& Número mayor que 0. \\ \hline
		\multirow{1}{*}{\textit{Enabled}} 		&  \checkmark	& Boolean \\ \hline
		\hline
	\end{tabular}
 	\caption{Resumen restrincciones formulario para \shippingEF.}
    \label{tab:dashboard:shipping:form:restrictions:shipping}
\end{table}

\begin{figure}[H]
	\centering
	\includegraphics[width=0.6\textwidth]{figuras/dashboard/shipping/form_shipping_add.png}
	\caption{Formulario para la creación de \shippingEF.}
	\label{figure:dashboard:shipping:form_shipping_add}
\end{figure}



\begin{figure}[H]
	\centering
	\includegraphics[width=0.6\textwidth]{figuras/dashboard/shipping/form_shipping_update.png}
	\caption{Formulario de actualización un \shippingEF.}
	\label{figure:dashboard:shipping:form_shipping_update}
\end{figure}


\subsubsection{Tarifas de \shippingEF y carros abadonados}

Aunque no es un tema que involucre a la finalidad de esta memoeria, es importante mencionar la importancia que tiene el factor \shippingEF dentro del abandono del carro de compra.
El desafio real que aparece cuando se desea abordar la estrategía de \shippingEF, es determinar la solución que impacte lo mínimo posible los margenes, pero que aún asi siga siendo atractivo para el cliente.
Y esto es algo en lo que se va querer estar en lo correcto. Estudios demuestran que el costo de \shippingEF es la principal razón del abandano de carros de compra (\refFigura{figure:dashboard:shipping:key_factor_shopping_cart_abandonment})\cite{online_forrester_consulting_smarter_stratefie_free_shipping}.

\begin{figure}[H]
	\centering
	\includegraphics[width=1\textwidth]{figuras/dashboard/shipping/key_factor_shopping_cart_abandonment.png}
	\caption{\shippingEF es un factor clave hacia el abandono de carros de compra \cite{online_forrester_consulting_smarter_stratefie_free_shipping}.}
	\label{figure:dashboard:shipping:key_factor_shopping_cart_abandonment}
\end{figure}

	% Dashboar payment
	%!TEX root = ../../../../memoria.tex
\subsection{\paymentsCOM}\label{cap:solucionImplementada:section:dashboard:subsection:payment}
	En establecimientos tradicionales \brickandmortar, un cliente ve un producto, lo examina, y entonces paga utlizando dinero, cheques, o tarjetas de crédito.
	% In traditional brick-and-mortar establishments, a customer sees a product, examines
	% it, and then pays for it by cash, check, or credit card (see Figure 6-1).

	En el mundo \ecommerceCOM, en la mayoría de los casos un cliente no ve físicamente el producto al momento de realizar la transacción, y el método de pago es realizado electrónicamente. Por lo tanto, temas de confianza y aceptación juegan un rol más importante en el mundo \ecommerceCOM que en el mundo de negocios tradicionales en lo que a sistemas de pago se refiere \cite{bidgoli2002electronic}.
	% In the e-commerce world, in most cases the customer does not physically see
	% the actual product at the time of transaction, and the method of payment is performed
	% electronically. Therefore, issues of trust and acceptance play a more important
	% role in the e-commerce world than in traditional businesses as far as payment
	% systems are concerned.

	Los sistemas de pago electrónico (\epsSiglasCOM, siglas en inglés) utilizan \hardwarePC y \softwarePC que permiten a los clientes pagar por productos y servicios en línea. Aunque estos sistemas podrían ser considerados inmaduros por los más críticos, no puede negarse el constante avance que viven periódicamente. El principal objetivo de un \epsSiglasCOM son incrementar eficiencia, mejorar seguridad, y lograr una mayor comodidad y facilidad de uso para los clientes \cite{bidgoli2002electronic}.
	% EPSs utilize integrated hardware and software systems that enable a customer
	% to pay for the goods and services online. Although these systems are in their infancy,
	% some significant progress has been made. The main objectives of EPS are
	% to increase efficiency, improve security, and enhance customer convenience and
	% ease of use. As shown throughout this chapter, there are several methods and
	% instruments that can be used to enable EPS implementation (see Figure
	% 6-2.)

	%Provide a Number of Payment Methods
	% 	It sounds obvious, but there are websites that offer only one payment method. However, data highlighted in an infographic from Milo shows that 56% of respondents expect a variety of payment options on the checkout page.
	Es importante agregar varios métodos de \paymentsCOM. Puede sonar obvio, pero existen métodos que aún ofrecen solo una manera de realizar un pago. Un estudio demuestra que el 56\% de usuarios espera una variedad de opciones de \paymentsCOM en la página de \checkoutCOM \cite{online_kissmetrics_easy_payment_process}.

	% While it’s not necessary – nor practical for that matter – to offer every conceivable payment method available, you’ll want to take a look at your target audience to see which payment methods they use.
	Aunque en la práctica es innecesario e impracticable ofrecer todos los métodos de pagos existentes, se deseará observar la preferencia del público para determinar cuáles son los métodos que ellos utilizan. Entonces se agregarán aquellos para capturar a la mayoría de los visitantes al \websiteINT \cite{online_kissmetrics_easy_payment_process}.

	En el contexto de esta memoria, solo se implementará \paypalNAME como servicio de pago.
	
	\subsubsection{\paypalNAME}

		\paypalNAME es un servicio global que mueve los montos de pago desde una tarjeta de crédito hacia el vendedor sin compartir la información financiera. \paypalNAME ofrece una variedad de productos y soluciones para aceptar pagos. De estos uno ha sido elegido para permitir el flujo completo de compra, pero siempre considerando la opción de agregar nuevos métodos en un futuro.

		% TODO: ver si alcanzo agregar el otro método
		%De estos, se han elegido dos los cuales han sido considerados adecuados para los requerimientos.

		%PayPal is a global service that moves the payment amount from your credit card to the merchant without sharing your financial information. %PayPal offers a variety of products and solutions for accepting payments. You can choose the solution that is best for your requirements, whether your goal is to get up and running as quickly as possible or to develop a fully customized payment experience.

		\subsection*{\PPPaymentProNAME}
			\PPPaymentProNAME es una solución customizable que permite a los vendedores mantener a los compradores dentro de su \websiteINT durante el proceso completo de \paymentsCOM y \checkoutCOM. Los vendedores pueden mantener sus propios sitios de \checkoutCOM personalizados y enviar transacciones a \paypalNAME, o es posible también tener un \hostCPT de \paypalNAME que tenga páginas de \checkoutCOM e incluso manejar la seguridad para las ventas y autorización. \PPPaymentProNAME puede aceptar pagos de \paypalNAME y  \paypalCreditNAME, al igual que con tarjetas de crédito de pago \cite{online_paypal_developer_acceptpayments}.
			% PayPal Payments Pro is a customizable solution that enables merchants to keep buyers on their website during the entire checkout and payment process. Merchants can host their own customized checkout pages and send transactions to PayPal, or they can have PayPal host the checkout pages and also manage security for sales and authorizations. PayPal Payments Pro can accept Paypal and PayPal credit payments, as well as credit and debit card payments. PayPal Payments Pro also includes an optimized mobile checkout experience. For details, see PayPal Payments Pro.

			Para el uso de este servicio, es necesaria la creación de una \paypalNAME \appINT. Toda la información necesaria se encuentra en la documentación para los desarrolladores \cite{online_paypal_developer_apps_credentials}, pero en resumen, lo relevante acá para lograr una configuración adecuada son 3 elementos:

				\begin{itemize}
					\item Existen dos ambientes de trabajo: \sandboxEnvPL y \liveEnvPL.
					\item \clientIDPayPal.
					\item \secretPayPal.
				\end{itemize}

			Tanto el \clientIDPayPal como el \secretPayPal dependen del ambiente en el cual me encuentro.

			Desde el \dashboardEF es posible acceder al menú configurable de \paypalNAME, el cual permite agregar esta información (ver \refFigura{figure:payment:paypal:payflow_pro:form}).

			\begin{figure}[H]
				\centering
				\includegraphics[width=0.6\textwidth]{figuras/payment/paypal/form_address.png}
				\caption{Formulario para agregar la configuración necesaria para el uso de \paypalPayFlowNAME.}
				\label{figure:payment:paypal:payflow_pro:form}
			\end{figure}

			El formulario de \paypalNAME se somete a las restricciones visibles en la \refTabla{tab:payment:paypal:payflow_pro:form}.

			\begin{table}[H]
			    \centering
				\begin{tabular}{ |l|c||l| }
					\hline Campo & Requerido & Restricción \\ \hline
					\multirow{1}{*}{\textit{Payflow enabled}} 	&  \checkmark 	& Boolean.\\ \hline
					\multirow{1}{*}{\textit{Client ID}} 		&  				& \\ \hline
					\multirow{1}{*}{\textit{Secret}} 			&  				& \\ \hline
					\multirow{1}{*}{\textit{PayFlow Mode}} 		&  \checkmark	& Boolean.\\ \hline
					\hline
				\end{tabular}
			 	\caption{Formulario para agregar la configuración necesaria para el uso de \paypalPayFlowNAME.}
			    \label{tab:payment:paypal:payflow_pro:form}
			\end{table}
			
