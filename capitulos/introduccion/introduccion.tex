%!TEX root = ../../memoria.tex
%%%%%%%%%%%%%%
%					%
%	INTRODUCCIÓN	%
%					%
%%%%%%%%%%%%%%
\chapter{Introducción}\label{cap:intro}

		%Historia de \ecommerce

		Una de las actividades más populares en la \webINT es comprar. La razón es porque tiene un encanto especial: puedes comprar en tiempo libre, en cualquier momento. Literalmente cualquiera puede tener páginas para mostrar sus productos y servicios, como ejemplo se encuentran \amazonNAME \cite{online_amazon_official}, \ebayNAME \cite{online_ebay_official} y \bestBuyNAME \cite{online_bestbuy_official}.

		La historia de \ecommerceCOM se remonta a la invención de la básica noción de \buyingCOM y \sellingCOM, electricidad, cables, computadores, \modems, y la \internetINT. \ecommerceCOM se hace posible en 1991 cuando la \internetINT estuvo disponible para uso comercial. Desde entonces miles de negocios se han establecido en sitios \webINT.

		En un inicio, el término \ecommerceCOM se refería al proceso de ejecución  de transacciones electrónicas comerciales con la ayuda de tecnologías lideres, tales como \electDataInterCOM y \electFundsTransCOM, las cuales dieron la oportunidad a los usuarios para intercambiar información de negocios y realizar transacciones electrónicas. La oportunidad para utilizar estas tecnologías surgió a finales de 1970s \cite{online_history_ecommerce} y permitió a los negocios de las compañías y organizaciones enviar documentación electrónica comercial.

		Aunque la \internetINT comenzó a ganar popularidad  entre el público general en 1994, tomó aproximadamente cuatro años desarrollar protocolos de seguridad (por ejemplo \httpNAME) y \dslModemNAME, los cuales permitieron un acceso rápido y conexiones persistentes a la \internetINT. Con el rápido crecimiento de la \internetINT, las personas comenzaron a considerar maneras de obtener un beneficio. Al mismo tiempo, las compañías entendieron que podrían utilizar \internetINT  como una forma de promover sus negocios a potenciales \customersCOM al rededor del mundo \cite{cook2015mobile}. Al principio, solo mostraban información sobre sus productos, pero los \customersCOM aun tenían que llamar a la compañía para realizar su \orderCommerce. Las compañías rápidamente comprendieron la necesidad de integrar la venta de sus productos y servicios sobre \internetINT a fin de alcanzar tantos \customersCOM como fuera posible. Con la mejora de la tecnología, las compañías tuvieron la posibilidad de no solo poner información de sus productos \online, además pudieron vender sus productos a los \customersCOM  también \cite{Maamar2003commerce}.
		%With the explosion of the Intemet, people looked for ways that they could benefit. At the same time companies were also looking at the Intemet as a way to help out their business. Companies realized that they could use the Intemet as a way to promote their business to potential customers all over the world (Venkatesh, Ramesh, & Massey, 2003). Companies saw the demand for the Intemet and decided to try to benefit fi-om this growth. At first, companies only posted information about their products on the Intemet, and consumers still had to call into to the company to place their orders. Companies soon realized that they needed to be able to sell their goods and services over the Intemet in order to reach the most customers that they could. With the improvements in technology, companies were then able to put not only put the information about their products online, they were actually able to sell their products to consumers as well (Maamar, 2003)

		%En 2000 un gran número de empresas comerciales en los Estados Unidos y Europa Occidental presentaron sus servicios en la \www. En ese entonces el significado de la palabra \ecommerceCOM fue cambiado. Las personas comenzaron a utilizar el término \ecommerceCOM como el proceso de compra de \itemsCOM y servicios disponibles en \internetINT utilizando conexiones seguras y servicios de pago electrónico. Aunque el colapso de \dotcom en 2000 dirigió a desafortunados resultados y muchas compañías \ecommerceCOM desaparecieron, los \retailers \brickandmortar reconocieron las ventajas del \ecommerceCOM e iniciaron la integración de tales características a sus sitios \webINT.

		Para finales de 2001, el modelo de negocio más grande de \ecommerceCOM, \btob, había ganado alrededor de \$700 billones en transacciones \cite{clark2011american}.

		% GRAFICO ECOMMERCE PERCENT SALES 
		%%%%%%%%%%%%%%%%%%%%%%%%%%%%%%%%%%%%%%%%%%%%%%%%%%%%%%%%%%%%%%%%%%%%%%%%%%%%%
%%%%%%%%%%%%%%%%%%%%%%%%%  Ecommerce Percent Sales  %%%%%%%%%%%%%%%%%%%%%%%%%
%%%%%%%%%%%%%%%%%%%%%%%%%%%%%%%%%%%%%%%%%%%%%%%%%%%%%%%%%%%%%%%%%%%%%%%%%%%%%

\begin{figure}[h!]
	\centering
	\includegraphics[width=0.7\textwidth]{figuras/ecommerce_percent.jpg}
    \begin{tikzpicture}[y=.7cm, x=.7cm,font=\sffamily]
        % Draw axes
        \draw [<->,thick] (0,7) node (yaxis) [above] {$y$}
            |- (11,0) node (xaxis) [right] {$x$};
        %ticks
       \foreach \x in {0,...,10}
            \draw (\x,1pt) -- (\x,-3pt)
            node[anchor=north] {\x};
        \foreach \y in {0,...,6}
            \draw (1pt,\y) -- (-3pt,\y) 
                node[anchor=east] {\y\%}; 
    \end{tikzpicture}
	\caption{\ecommerce como porcentaje de las ventas totales en Estados Unidos \cite{online_total_sales_2000_2012}}
	\label{figure:ecommerce_percent_sales}
\end{figure}

		Acorde a toda información disponible, las ventas \ecommerceCOM continuaron creciendo en los siguientes años y en 2012 las ventas \ecommerceCOM representaron el 5.2\% de las ventas totales en \usaNAME. Como se puede observar en \refFigura{figure:ecommerce_percent_sales}.

		\ecommerceCOM tiene muchas ventajas sobre tiendas \brickandmortar y catálogos de venta por correo. Los consumidores pueden fácilmente buscar a través de una \dataBasesDB muy extensa de productos y servicios. Pueden ver los precios reales, definir una orden de compra en varios días y enviar un correo como un \wishlist esperando que alguien pague por sus productos seleccionados \cite{online_whishList_important}. Los clientes pueden comparar precios con un simple \click del \mousePC y comprar los productos seleccionados al mejor precio.

		Proveedores \online, en sus turnos, también tienen claras ventajas. La \webINT y sus motores de búsqueda proveen una manera para encontrar clientes sin campañas de publicidad costosas. Incluso tiendas \online pequeñas pueden alcanzar mercados globales. 

		La tecnología \webINT también permite realizar un seguimiento sobre las preferencias de los clientes para ofrecer una comercialización personalizada.

		%La historia de \ecommerceCOM is impensable sin Amazon e Ebay los cuales estuvieron entre las primeras compañías que permitían transacciones electrónicas. Gracias a sus fundadores ahora tenemos un sector considerable de \ecommerceCOM y disfrutar de comprar y vender gracias a la Internet. Actualmente hay 5 de los mas grandes y mas famosos \textit{worldwide internet retailers}: Amazon, Dell, Staples, Office Depot y Hewlett Packard. De acuerdo a las estadísticas, las categorías de productos mas vendidos en \textit{Worl Wide Web} son música, libros, computadores, artículos de oficina y otros dispositivos electrónicos.
		%
		%Amazon.com, Inc. es uno de las mas famosas compañías \ecommerceCOM  para vender productos sobre la Internet. Después del colapso \textit{dot-com} Amazon perdió su posición de modelo de negocio exitoso, sin embargo, en 2003 la compañía hizo su primer año con utilidades el cual fue el primer paso para el desarrollo futuro.
		%
		%Al principio Amazon.com fue considerado como una tienda \textit{online} de libros, pero con el tiempo una variedad de productos electrónicos fueron agregados, software, DVDs, juego de video, CDs de música, MP3s, prendas de vestir, calzado, productos de salud, etc. El nombre original de la compañía fue Cadabra.com, pero rápidamente después que se volviera popular Internet Bezos decidió cambiar el nombre de su negocio a “Amazon” después del río voluminoso mas grande del mundo. En 1999 Jeff Bezos fue nombrado como la persona del año por Time Magazine en reconocimiento al éxito de la compañía. Aunque la sede principal de la empresa se encuentra en USA, WA, Amazon ha establecido sitios \textit{web} separados en otros países tales como United Kingdom, Canada, France, Germany, Japan, y China. La compañía apoya y opera \textit{retail web sites} para muchos negocios famosos, incluyendo Marks \& Spencer, Lacoste, la NBA, Bebe Stores, Target, etc.
		%
		%Amazon es uno de los primeros negocios \ecommerceCOM en establecer un programa de marketing para los afiliados, y actualmente la compañía obtiene cerca del 40\% de sus ventas desde afiliados y vendedores de terceras partes que lista y vende productos en el sitio \textit{web}. En 2008 Amazon penetro en el cine y actualmente esta patrocinando la película \textit{“The Stolen Child”} con \textit{20th Century Fox}.
		%
		%Acorde a las investigaciones en 2008, el dominio Amazon.com a traído cerca de 615 millones de clientes cada año. La característica mas popular del sitio \textit{web} es el \textit{review sistem}, i.e., la habilidad de los visitantes de presentar sus \textit{reviews} y \textit{rate} cualquier producto en un \textit{raiting scale} de uno a 5 estrellas. Amazon.com es también \textit{well-know} por su \textit{clear and user-friendly} avanzado sistema de búsqueda que permite a los visitantes buscar por \textit{keywords} en el texto completo de muchos libros en la base de datos.
		%
		%Otra compañía ha contribuido mucho en el proceso de desarrollo de \ecommerceCOM es Dell Inc., una compañía americana posicionada en Texas, que se sitúa en el tercer lugar de ventas de computadoras después de Hewllett-Packard y Acer.
		%
		%Lanzado en 1994 como una pagina estática, Dell.com ha realizado rápidos pasos, y para finales de 1997 fue la primera compañía en lograr el \textit{record} de un millón de ventas \textit{online}. Su única estrategia de ventas sobre la \textit{World Wide Web} sin \textit{retail outlets} y sin intermediarios ha sido admirado por un sin fin de clientes e imitado por un gran numero de \ecommerceCOM \textit{businesses}. El factor de éxito de Dell es que Dell.com permite a los clientes elegir y controlar, i.e., visitantes pueden explorar el sitio y ensamblar PC pieza por pieza eligiendo el mas mínimo componente basado en sus presupuestos y requerimientos. De acuerdo a las estadísticas, aproximadamente la mitad de las ganancias de las compañías proviene desde su sitio \textit{web}.
		%
		%En 2007, Fortune magazine \textit{ranked} Dell como la \textit{34th-largest} compañía en \textit{Fortune 500 list}, y \textit{8th} en su \textit{Top 20 list} anual de las mas exitosas y admiradas compañías en USA en reconocimiento al \textit{bussiness model} de la compañía.

		%La historia de \ecommerceCOM es nueva, un mundo virtual que esta evolucionando de acuerdo a las ventajas del cliente. Es un mundo que todos construyen en conjunto ladrillo por ladrillo, estableciendo una base segura para las nuevas generaciones.

		\subsection{\ecommerceCOM en la actualidad}

			% *************************  TODO INIT *************************
			En la actualidad, \ecommerceCOM es una experiencia destacable. Transformó las compras tradicionales más allá de lo reconocible. La experiencia es mucho mejor que cualquier otra manera de comprar, seduciendo a una gran cantidad de \ecommerceCOM \lovers.
			% *************************  TODO END *************************

			Si hace algunos años \ecommerceCOM fue una palabra de moda, ahora se ha convertido en la orden del día. Al parecer las personas compran literalmente en cualquier parte, en sus lugares de trabajo, durante el almuerzo, en las horas punta cuando no hay nada más por hacer salvo encender el computador y comenzar a navegar.

			En el presente, \ecommerceCOM ha ganado tanta popularidad debido a que su tecnología subyacente está evolucionando a pasos agigantados. Incluso se está ofreciendo \textit{"sentir"} el producto con un \mousePC \textit{3D} para comprender mejor su forma, tamaño y textura. ¿Para qué salir cuando todo lo que se debe hacer es realizar un pedido, elegir la forma de envío, pararse y esperar hasta que la orden sea entregada en la puerta de la casa?

			% *************************  TODO INIT *************************
			\ecommerceCOM hoy ofrece tanta comodidad que incluso las tiendas convencionales han encendido las alarmas. Aunque, cada uno está de acuerdo en que hay un gran camino para que \ecommerceCOM reemplace las tiendas, la posibilidad existe que ocurra en el futuro. \ecommerceCOM del cual somos actualmente testigos trae tanta aventura a nuestras vidas que es disfrutado por toda la comunidad \online.
			% *************************  TODO END *************************

			En la actualidad \ecommerceCOM tiene algunos inconvenientes, sin embargo, muchos consumidores están dispuestos a aguantar estas desventajas, ya que confían en el mundo \online y desean que sea un mejor lugar.

			\ecommerceCOM refleja lo que hemos creado desde un inicio para el comercio electrónico \online. Fue creado por nosotros y pensado para nosotros.

		\subsection{\ecommerceCOM en el futuro}

			% *************************  TODO INIT *************************
			Expertos predicen un futuro prometedor para \ecommerceCOM. En un presumible futuro \ecommerceCOM se confirmará por sí misma como una herramienta importante para las ventas. El éxito de \ecommerceCOM se convertirá en una noción inseparable de la \webINT, ya que \eshopping se está volviendo cada vez más y más popular y natural. Al mismo tiempo la competitividad en el mundo de los servicios \ecommerceCOM intensificará su crecimiento. Así la tendencia de que prevalecerá \ecommerceCOM será el crecimiento de las ventas por \internetINT y la evolución.
			% *************************  TODO END *************************

			Cada año el número de ofertas de \ecommerceCOM crece enormemente. El volumen de ventas de las tiendas \online  es más que comparable con esos \brickandmortar. Y la tendencia va a continuar \cite{online_growth_ecommerce}, porque hay mucha gente \textit{limitada} por las obligaciones laborales y de los hogares, mientras que \internetINT permite ahorrar mucho tiempo y dinero al permitir  elegir los productos a los mejores precios. 
			%Hoy en día el auge de las ventas por \internetINT es la base del magnífico futuro \ecommerceCOM.

			La tendencia de \textit{cantidad a calidad} del \ecommerceCOM se está haciendo cada vez más evidente, ya que \internetINT ha incluido el factor geográfico de la venta. Así que no importa si su tienda se encuentra en \newYorkNAME, \londresNAME o en una pequeña ciudad. Para sobrevivir, los comerciantes tendrán que adaptarse rápidamente a las nuevas condiciones. Para atraer a más clientes, los propietarios de las tiendas \online tendrán no solo que incrementar el número de servicios disponibles, además deberán poner más atención a elementos como \designQA atractivo, \userfriendliness, presentación atractiva; tendrán que oportunamente emplear tecnología moderna para que sus negocios sean parte del futuro del \ecommerceCOM.

			% *************************  TODO INIT *************************
			Claramente, aquellos que adquieren \estores antes, tienen mejores oportunidades  para el éxito y prosperidad, aunque un sitio \ecommerceCOM por sí mismo no garantiza nada. Solo una apropiada solución \ecommerceCOM en combinación con \emarketing y publicidad pueden asegurar éxito.
			% *************************  TODO END *************************

			% *************************  TODO INIT *************************
			\ecommerceCOM está cambiando fundamentalmente la economía y la manera en que los \businessCOM son conducidos. \ecommerceCOM obliga a las compañías a encontrar nuevas maneras para expandir los mercados en los que compite, para atraer y retener \customersCOM adaptando los productos y servicios para sus necesidades, y para reestructurar sus procesos de \businessCOM para entregar productos y servicios de forma más eficiente y efectiva. Sin embargo, a pesar de los rápidos y sostenidos desarrollos de \ecommerceCOM, muchas empresas que realizan negocios electrónicos aún permanecen en la fase de inversión y creación de marca. Muchos \ebusinessCOM se han centrado en el atractivo visual y la facilidad de uso del \websiteINT como la manera para aumentar la base de \customersCOM. Sin embargo, tal como \ebusinessCOM cambia su enfoque de construir una base para el \customerCOM para incrementar el \revenueQA y la rentabilidad, deben re-evaluar sus sus estrategias para proveer uuna clara rentabilidad \cite{shin2001strategies}.
			%E-commerce is fundamentally changing the economy and the way business is conducted. E-commerce forces companies to find new ways to expand the markets in which they compete, to attract and retain customers by tailoring products and services to their needs, and to restructure their business processes to deliver products and services more efficiently and effectively. However, despite rapid and sustained development of e-commerce, many companies doing e-business are still in the investment and brand-building phase and have yet to make a profit (Zwass 1998). Many e-businesses (or Internet companies) have focused on the visual attractiveness and ease of use of their Web sites as the primary method of increasing their customer base. However, as e-businesses shift their focus from building a customer base to increasing revenue growth and profitability, they should re-evaluate their current business strategies, if any, and develop strategies that provide a clear path to profitability. 
			% *************************  TODO END *************************

			En el mundo empresarial, el tamaño importa, pero ni siquiera eso garantiza un futuro en un mercado de constante cambio y con una competencia feroz. Innumerables son las empresas que murieron en el intento, entre las cuales podemos destacar: \sega, \kodak, \daewoo, \nokia, \blockbuster, entre otros.

			En la actualidad existe una gran variedad de \frameworksPC \openSourcePC disponibles que permiten desarrollar una gran variedad de soluciones \ecommerceCOM. Todas estas opciones tienen grandes comunidades que las respaldan, así como muchos usuarios satisfechos \citeAllFrameworks. 

			Todas estas opciones fueron construidas en un escenario en donde el poder de procesamiento en los clientes era limitado, pero en estos días un solo \iphone tiene un poder de cómputo superior a la mayoría de los súper computadores en los inicios de la \webINT. 

			Después de años de avance en computadores personales, tecnologías creativas han surgido, y los \webINT \browsersINT han evolucionado para mantenerse al día. En la actualidad, la \webINT ha madurado alcanzando una serie de características que permiten aplicaciones enriquecidas que mejoran las experiencias de los usuarios. Es entonces factible desarrollar \frameworksPC \ecommerceCOM mejores tanto del punto de vista del usuario final como de los desarrolladores.

	\section{Motivación}\label{cap:intro:motivacion}

		% *************************  TODO INIT *************************
		Los \online \retailers están en la cúspide de una forma totalmente nueva de hacer negocios. Ellos tienen una oportunidad única para obtener ventajas competitivas significativas en sus respectivos mercados, siempre que entreguen consistentemente una experiencia grata para el consumidor y permitan características de comercio \multichannel únicos antes que sus competidores. El éxito dependerá en perfeccionar los esfuerzos para abordar las experiencias de clientes centrados en el usuario, reducir el enfoque a los programas más valiosos, y eligiendo estrategias de tecnología adecuada que permitan a los equipos internos ofrecer experiencias escalables optimizadas. Las áreas emergentes a observar son \realTimeINT, \retail \analytics, incorporar \socialnetwork en \ecommerceCOM, el continuo éxito de aplicaciones móviles, y herramientas que permitan a \retailers escalar experiencias que dirijan de manera más efectiva el \merchandising.
		%Online retailers are on the cusp of a totally new way of doing business. They have a unique opportunity to gain a significant competitive advantage in their respective markets, provided they consistently deliver a consumer-friendly experience and enable unique multichannel commerce behaviors before their competitors. Success will depend on honing efforts to address user-centric customer experiences, narrowing the focus to the most-valuable programs, and electing the right technology strategy that will enable internal teams to deliver optimized scalable experiences. Areas to watch are the emergence of real-time retail analytics, social-network enabled commerce, the continued success of mobile applications, and tools that will enable retailers to scale “always-targeted” experiences that target merchandising more effectively.

		% *************************  TODO END *************************

		Existe una amplia variedad de escenarios en donde se encuentran soluciones \ecommerceCOM o similares con la potencialidad de ser mejorados permitiendo entrar al mercado con ventajas competitivas. A continuación se describen variadas situaciones que tienen gran potencial latente.

		\begin{itemize}
			\item
				\textbf{\shoppingCart}: software \ecommerceCOM en un servidor \webINT que permite a los visitantes del \websiteINT seleccionar productos para eventualmente comprarlos.
				%is a piece of e-commerce software on a web server that allows visitors to an Internet site to select items for eventual purchase
			
			\item
				\textbf{Reclamos ciudadanos}: Un sistema para realizar y gestionar reclamos que estén relacionados principalmente con temas sociales con el fin de alertar rápidamente a los organismos públicos responsables. Ejemplos de uso serían: calles en mal estado, semáforos dañados, semáforos mal sincronizados, cruces peligrosos para peatones, aceras en mal estado, lugares de ruido excesivo, zonas de robos comunes, plazas en mal estado, accesos cerrados a playas. La idea básica es que un usuario pueda crear contenido (quejas), permitiendo a otros usuarios apoyarlas, para informar oportunamente a las autoridades responsables.
			
			\item 
				\textbf{Reservas de horas médicas:} Un sistema para gestionar las horas médicas que muestre sugerencias de acuerdo a la distancia que se encuentran los especialistas desde la \geoPositionCPT del sistema que realiza la solicitud. 
			
			\item
				\textbf{Consultas de libros en una biblioteca:} Un sistema para consultar la disponibilidad, la cantidad, la posibilidad de reservarlo, e incluso cuándo será devuelto.

			\item
				\textbf{Catálogo de moteles}: Un catálogo de moteles de acuerdo a la \geoPositionCPT pueden ser listados para determinar las mejores opciones disponibles. Adicionalmente se podría mostrar los horarios disponibles de las habitaciones, e incluso poder generar reservas y pagar la habitación. De esta manera se optimiza el tiempo de los clientes, y los recursos de los proveedores.
			
			\item
				\textbf{Catálogo de medicamentos}: Un catálogo completo de los medicamentos oficialmente aceptados podrían ser listados para entregar información relevante de acuerdo a las dosis así como de efectos adversos, si es necesaria una receta, etc. Otra cosa interesante es dar la opción de mostrar todos los remedios genéricos que existen en el mercado. El escenario ideal sería además mostrar el precio de estos, así como la distancia a las droguerías en donde el producto se encuentra disponible.
			
				%\item \textbf{Reserva de horas en Registro Civil}: Al menos en Chile, realizar un tramite en el registro civil es sinónimo de una perdida de tiempo absurda solo para ser atendido. Si la reserva de hora puede ser realizada desde cualquier sitio, el sistema se volvería drásticamente más eficiente.
			
			\item
				\textbf{Catálogo de Fiestas}: Un catálogo con las fiestas disponibles tanto a la brevedad posible como en el futuro. Que muestre información relevante. Y que permita la compra de entradas.
			
			\item
				\textbf{Gestor de detalle de cuenta en restaurante}: Permite gestionar el detalle de la cuenta de un restaurante. La idea básica es tener una mejor visión tanto de mis órdenes como las del total de la mesa en la cual me encuentro, evitando así problemas en relación con el ajuste total de la cuenta, sorpresas en el detalle del consumo, cobros erróneos, etc. Además de permitir prepagos que no necesariamente se realizarán al final de la reunión (posibles clientes que se retiren antes que el grupo en el cual se encuentran).
			
		\end{itemize}

		Todas estas situaciones tienen algo en común, y es que los usuarios desean consultar información para tomar una decisión. El sistema idealmente permite contrastar datos entre diferentes proveedores de servicios similares, así como de aportar con información clara y relevante, para que futuros usuarios puedan tomar una decisión aún más informados.

				Como se puede concluir, el modelo de negocio de estas situaciones es en su mayoría el mismo, solo cambiando la presentación para hacerlo \adhoc a cada situación \citeAllFrameworks.

		En la actualidad, existe una gran variedad de \frameworksPC disponibles que gracias a sus funcionalidades genéricas, agrupadas selectivamente, permiten  desarrollar estas y otras soluciones específicas agregando código escrito por un desarrollador. Sin embargo, estas herramientas fueron adoptadas cuando muchos de los conceptos actuales en los que se mueve la \internetINT no existían, como por ejemplo:
		% el desarrollo de aplicaciones \realTimeINT, \reactive las cuales permiten deseables \featuresCPT en un \frameworkPC de estas características; aunque pueden tener soporte, no fueron construidas considerando los \devicesINT \mobilesINT .

		\begin{itemize}
			\item
				Desarrollo de aplicaciones \realTimeINT, \reactive, las cuales permiten deseables \featuresCPT en un \frameworkPC de esta naturaleza.
			\item
				Soporte completo para desarrollo en \devicesINT \mobilesINT, permitiendo nuevas ventajas para \ecommerceCOM \cite{cook2015mobile}.
		\end{itemize}

		%Después de años en avances en computación personal, tecnologías creativas han impulsado el desarrollo de aplicaciones \webINT permitiendo características similares a las aplicaciones nativas.
		%After years of advances in personal computing, creative technologists have pushed the web to its limits, and web browsers have evolved to keep up. Now, the Web has matured into a fully-featured application platform, and fast JavaScript runtimes and \htmlfive standards have enabled developers to create the rich apps that before were only possible on native platforms.

		%quiero transmitir la idea que los ecommerce actuales se han desactualizado. Por que fueron creados con herramientas que no consideraban muchas de las necesidades actuales.

		Se propone desarrollar un \frameworkPC \freePC \openSourcePC utilizando tecnología \webINT que permita crear aplicaciones enriquecidas con características no presentes en los \frameworksPC \ecommerceCOM \freePC \openSourcePC disponibles en la actualidad.
		%http://nerds.airbnb.com/isomorphic-javascript-future-web-apps/

		%Se propone entonces desarrollar un \frameworkPC con tecnología de punta creadas considerando la \internetINT en la que se vive en la actualidad, permitiendo no solo desarrollar funcionalidades que antes no eran posibles, y solucionar problemas que estaban surgiendo; sino que ademas entregan mayor flexibilidad en el desarrollo de soluciones específicas, menor cantidad de recursos humanos y menor tiempo de desarrollo.

	\section{Alcances y objetivo general}\label{cap:intro:alcances}

		Implementar un \frameworkPC \ecommerceCOM utilizando tecnología \freePC \openSourcePC que permita (en comparación a los \frameworksPC actuales):  resultados más rápidos, ideal para \prototypesCPT y \mvpSiglasCOM; una solución completa para crear \featuresCPT para \serversAS \cite{online_ecommerce_solution_requires}, \browsersINT y \devicesINT \mobilesINT; \builtINPL con capacidades \realTimeINT para minimizar esfuerzo en \developmentPC; e integración de herramientas de desarrollo para hacer que \setupCPT, \developmentPC, y \deploymentCPT sea extremadamente rápido. 

		%Implementar un \frameworkPC \ecommerceCOM utilizando tecnología \freePC \openSourcePC que permita la desarrollar aplicaciones enriquecidas \fullFeatured, \realTimeINT, \reactive, mayor flexibilidad y/o disminución de la complejidad, en comparación a las \frameworksPC actuales, en el desarrollo de soluciones específicas. Lo que implica mayor diversidad de aplicaciones, menor cantidad de recursos humanos y menor tiempo de desarrollo.

	\section{Objetivos específicos}\label{cap:intro:objetivos}

		%Dentro de los objetivos específicos se encuentran:
		\begin{itemize}
			\item Especificar los requerimientos del diseño del \frameworkPC.
			\item Escoger la tecnología más adecuada para el desarrollo del proyecto.
			\item Determinar la arquitectura adecuada para el desarrollo del \frameworkPC.
			\item Implementar la solución.
			\item Desarrollar una solución utilizando \frameworkname, que permite contrastar las potencialidades con respecto a \frameworksPC existentes.
			%\item Validar \frameworkPC, desarrollando al menos 3 soluciones específicas.
			%\item Determinar, en base a las 3 soluciones desarrolladas, si es efectivo que \frameworkPC disminuye la complejidad de desarrollo. 
		\end{itemize}

		% CARACTERISTICAS DESEABLES
		%\section{Caracteristicas deseables}\label{cap:intro:alcance}

\begin{table}[h!]
    \tiny
   
\begin{tabular}{ |L{0.6\paperwidth}|L{0.1\paperwidth}|}
\hline
	\task &
	Tiempo
	
\\ \hline
	\textbf{ Agregar \itemCommerce}: Agregar un \itemCommerce a la tienda.&
	
\\ \hline
	\textbf{ Remover \itemCommerce}: Remover un \itemCommerce de la tienda.&
	
\\ \hline
	\textbf{ Agregar \itemCommerce}: &
	
\\ \hline
	\textbf{ Agregar \itemCommerce}: &
	
\\ \hline
	\textbf{ Agregar \itemCommerce}: &
	
\\ \hline
	\textbf{ Agregar \itemCommerce}: &
	
\\ \hline
	\textbf{ \textit{\dragdrop} manejo de variedad}: Organizar el orden de la variedad de \itemCommerce en la tienda con \textit{\dragdrop}.&
	%\textbf{ Drag \& Drop Variant Management:} Arrange order of product's variants in your shop with drag and drop.&
		
\\ \hline
	\textbf{ \dragdrop \merchandising}: Organizar el orden de los productos en la tienda con \textit{\dragdrop}.&
	%\textbf{ Drag \& Drop Merchandising:} Arrange order of products in your shop with drag and drop.&
		
\\ \hline
	\textbf{Independencia del \device}: optimizar la experiencia para todas las plataformas: \mobile, \tablet y \desktop \devices.
	%\textbf{ Device Agnostic:} Optimized experience for all mobile, tablet, and desktop devices.&
		
\\ \hline
	 \textbf{Edición del campo \inline}: Agregar y editar contenido de la tienda \clicking cualquier \textfield.&
	% \textbf{ Inline Field Editing:} Add and edit your shop's content by clicking any text field.&
		
\\ \hline
	\textbf{ \googleanalytics}: \outofthebox \googleanalytics \event \tracking.&
	%\textbf{ \googleanalytics}: Out of the box Google Analytics event tracking.&
		
\\ \hline
	\textbf{ Integración de \paypal}: Soportar la opción de utilizar \paypal \checkout.&
	%\textbf{ PayPal Integration:} Supports the ability to use PayPal checkout.&
		
\\ \hline
	\textbf{\realtime \itemupdates}: Cambios realizados a la tienda son instantáneamente observados por los compradores (sin refrescar la página).&
	%\textbf{ Real-time Reactive:} Changes made to your shop are instantly seen by your shoppers (without page reloads).&
		
\\ \hline
	\textbf{Clonar productos existentes}: Crear nuevos productos clonando cualquier producto existente.&
	%\textbf{ Clone Existing Products:} Create new products by cloning any existing products.&
		
\\ \hline
	\textbf{Múltiples imágenes por variante de producto}: Agregar múltiples imágenes por opción de producto.&
%	\textbf{ Multiple Images Per Product Variant:} Add multiple product images per product option.&
		
\\ \hline
	\textbf{ Recursive Tag Taxonomy:} Uses recursive tag taxonomy para categorizar.&	
	%\textbf{ Recursive Tag Taxonomy:} Uses recursive tag taxonomy for categorization.&
		
\\ \hline
	 \textbf{ Clone Product Variants:} Create new product variants by cloning any existing product variant.&
	
\\ \hline
	\textbf{ Fully Open Source:} Node.js and Meteor. Completely Open source. All day. Every day.&
	
\\ \hline
	\textbf{ Quantity per Variant:} Inventory support by product variant.&
	
\\ \hline
	\textbf{ Published or Hidden Status:} Ability to have products in a "draft" state before publishing.&
	
\\ \hline
	\textbf{ Detalle de productos}: permite detalle de productos en \keyvalue \listhighlevel .&
%	\textbf{ Product Details:} Supports additional product details in key/value list.&
	
\\ \hline
	\textbf{ SEO Hashtags:} Uses social media hashtags for product urls for simple SEO+social media \tracking.&
	
\\ \hline
	\textbf{ \onepage \checkout:} \checkout todo en \onepage.&
	%\textbf{ One Page \checkout:} Checkout all done on one page.&
	
\\ \hline
	\textbf{ \shop \analytics:} Integrar \tracking \framework para integrar cualquier \analytics \system.&
	%\textbf{ Shop Analytics:} Integrated tracking framework for integration to any analytics system&
	
\\ \hline
	\textbf{ \dockerio listo}: \build, \ship y \run en cualquier lugar.&
	%\textbf{ Docker.io Ready:} Build, Ship and Run anywhere.&
	
\\ \hline
	 \textbf{ Backorders:} Shoppers can purchase products even when quantity runs out.&
	
\\ \hline
	\textbf{ i18n and l10n:} Internationalization and localization to translate and localize all content for the world.&
	
\\ \hline
	\textbf{ Variant Management:} Easily manage product options (e.g. size/color) and related product photos.&
	
\\ \hline
	\textbf{ App Gallery:} Select from a gallery of apps to extend your shop by adding your favorite tools.&
	
\\ \hline
	\textbf{ Social Media Integration:} Integrated custom product social media messaging (FB, Twitter, Pinterest, Instagram).&
	
\\ \hline
	\textbf{ Custom Domains:} Use your own domain names.&
	
\\ \hline
	\textbf{ Flat Rate Tax Management:} Manage tax rules for your store.&
	
\\ \hline
	\textbf{ Additional Payment Methods:} Support for more payment methods (providers TBD).&
	
\\ \hline
	\textbf{ Global Shipping:} Ship your products around the globe (carriers TBD).&
	
\\ \hline
	\textbf{ User Management:} Invite users and grant permissions.&
	
\\ \hline
	\textbf{ Email Templates:} Manage email templates for your shop.&
	
\\ \hline
	\textbf{ Hero Manager:} Inline management for hero sections on your shop.&
	
\\ \hline
	\textbf{ Search:} Auto-suggest, keyword product search.&
	
\\ \hline
	\textbf{ Braintree Integration:} Braintree Integration.&
	
\\ \hline
	\textbf{ Stripe Integration:} Use Stripe for payments.&
	
\\ \hline
	\textbf{ Native Mobile Apps:} Build and deploy iOS, Android native Reaction apps.&
	
\\ \hline
	\textbf{ Multi-Currency Support:} Support for additional currencies beyond US Dollars.&
	
\\ \hline
	\textbf{ RTL Localization:} Right to left language support.&
	
\\ \hline
	\textbf{ Revision Control:} Revision control with rollback for all edits.&
	
\\ \hline
	\textbf{ History Logging:} Full insight to all actions performed.&
	
\\ \hline
	\textbf{ Cash on Delivery Payments:} Cash on delivery payment methods.&
	
\\ \hline
	\textbf{ API:} Support for both HTTP API and Meteor DDP.&
	
\\ \hline
	\textbf{ Product Inheritance:} Manage product pricing, promotions, visibility through parent-child-clone inheritance.&
	
\\ \hline
	\textbf{ Coupon Codes:} Coupon management and tracking for discounts.&
	
\\ \hline
	\textbf{ Returns and Refunds:} Track, manage, and analytics on returns and exchanges.&
	
\\ \hline
	\textbf{ Multi-Vendor:} Multiple vendors with review, publish, drop shipping.&
	
\\ \hline
	\textbf{ Flexible Tax Management:} Manage and customize tax rules.&
	
\\ \hline
	\textbf{ Subscription Products:} Subscription based product types.&
	
\\ \hline
	\textbf{ Order Entry and Editing:} Adminstrator addition and editing of orders.&
	
\\ \hline
	\textbf{ Embed Social Content:} Embed reviews, tweets, and other social content.&
	
\\ \hline
	\textbf{ Bitcoin Integration:} Ability to accept Bitcoin payments in your shop.&
	
\\ \hline
	\textbf{ Amazon Payments Integration:} Use Amazon for payments.&
	
\\ \hline
	\textbf{ Promotions:} Ability to manage and track promotions by channels, events, and more.&
	
\\ \hline
	\textbf{ Google Wallet Integration:} Use Google Wallet for payments.&
	
\\ \hline
	\textbf{ Shipwire Integration:} Ability to use Shipwire for order fulfillment.&
	
\\ \hline
	\textbf{ ShipStation Integration:} Ability to use ShipStation for order fulfillment.&
	
\\ \hline
	\textbf{ MailChimp Integration:} Use MailChimp to collect and manage emails.&
	
\\ \hline
	\textbf{ Actionable Analytics:} Data driven product presentation, and performance analysis.&
	
\\ \hline
	\textbf{ Hotkeys:} Establish shortcuts for regular tasks to quickly trigger a Reaction action.&
	
\\ \hline
	\textbf{ Theme Gallery:} Select from a gallery of themes to change the design and experience of your shop.&
	
\\ \hline
	\textbf{ Import from Squarespace:} Import your product catalog from Squarespace into Reaction.&
	
\\ \hline
	 \textbf{ Import from Shopify:} Import your product catalog from Shopify into Reaction.&
	
\\ \hline
	\textbf{ Import from Magento:} Import your product catalog from Magento into Reaction.&
	
\\ \hline
	\textbf{ Import from Spree Commerce:} Import your product catalog from Spree Commerce into Reaction.&
	
\\ \hline
	\textbf{ Import from Big Commerce:} Import your product catalog from Big Commerce into Reaction.&
	
\\ \hline
	&
	
\\ \hline
	&
				
\\ \hline
\end{tabular}
    \caption{ Carta gant}
    \label{tab:task_proyect}
\end{table}

%TODO remover este codigo inecesario
%\section{Estructura de la memoria}\label{cap:intro:estructura}
%La estructura utilizada en este documento para exponer el trabajo realizado es la siguiente:
%
%\begin{itemize}
%	\item \textbf{Capítulo \ref{cap:intro}. \nameref{cap:intro}:} Corresponde a la descripción del tema, la motivación de éste y los alcances y objetivos del trabajo realizado.
%	
%	\item \textbf{Capítulo \ref{cap:antecedentes}. \nameref{cap:antecedentes}:} Corresponde a la revisión bibliográfica o antecedentes. En este capítulo se explican los conceptos necesarios para la comprensión y contextualización del trabajo.
%	
%	\item \textbf{Capítulo \ref{cap:conclusiones}. \nameref{cap:conclusiones}:}	Se enumeran las conclusiones del trabajo realizado y se proponen trabajos a realizar en el futuro.
%\end{itemize}