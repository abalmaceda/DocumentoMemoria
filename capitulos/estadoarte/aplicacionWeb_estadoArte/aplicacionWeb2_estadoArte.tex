
\section{Aplicaciones \webINT}\label{cap:estadoArte:section:web_app}

\subsection{\singlePageAppINT}

\subsection{Aplicaciones \isomorphicAS}

\section{El futuro de las aplicaciones \webINT}

Cuanto más organizaciones se sientan cómodas \runningCPT \nodejsNAME en \productionPC, es inevitable que cada vez más y más aplicaciones \webINT comiencen a compartir código entre el \clientAS y el \serverAS. Es importante recordar que \isomorphicAS \javaScriptNAME es una gama - puede empezar solo compartiendo \templatesAS, progresando hasta convertirse
%As more organizations get comfortable running Node.js in production, it’s inevitable that more and more web apps will begin to share code between their client and server code. It’s important to remember that isomorphic JavaScript is a spectrum — it can start with just sharing templates, progress to be an entire application’s view layer, all the way to the majority of the app’s business logic. Exactly what and how JavaScript code is shared between environments depends entirely on the application being built and its unique set of constraints.

%Nicholas C. Zakas has a nice description of how he envisions apps will begin to pull their UI layer down to the server from the client, enabling performance and maintainability optimizations. An app doesn’t have to rip out its backend and replace it with Node.js to use isomorphic JavaScript, essentially throwing out the baby with the bathwater. Instead, by creating sensible APIs and RESTful resources, the traditional backend can live alongside the Node.js layer.