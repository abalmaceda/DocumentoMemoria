
\subsection{\keyelements de un sitio \ecommerceCOM \cite{inbook_ecommerce_keyelements}}

%TODO
\subsubsection{Domain y Hosting}

\subsubsection{Diseño}
El diseño del sitio es el principal elemento en determinar si el visitante se queda o se va.

%TODO
\subsubsection{\usabilityQA}

%TODO
\subsubsection{conversion}

%TODO
\subsubsection{\checkoutCOM}

%TODO
\subsubsection{Taking Payments}

%TODO
\subsubsection{Product Pages}

%TODO
\subsubsection{On-Site SEO (Search Engine Optimization)}

%TODO
\subsubsection{Content Management System \& Automation}

%TODO
\subsubsection{Key \ecommerceCOM Features}

%TODO
\subsubsection{Operation \& Order Management}

%TODO
\subsubsection{Information Pages}


\subsubsection{\security, \trust \& Testimonials}

\trust y \familiarity influencian \ecommerceCOM, como insinúa \luhmanntheory. Especialmente los datos muestran que tanto \trust en un \consumer \internetINT y \familiarity con el \seller y sus procedimientos influencian dos aspectos distintos de \ecommerceCOM en sitios de venta de libros: \inquiry y \purchase. La influencia de la  \familiarity y \trust son especialmente fuertes en intenciones  de las personas para \purchase. Segundo, los datos muestran que \trust y \familiarity son claramente diferentes construcciones, y la \trust es significativamente afectada por \familiarity, y no solo por la social disposición de la gente para \trust. El modelo de investigación muestra que  \ecommerceCOM puede ser evaluado en el contexto de un ambiente social complejo basado en \luhmanntheory. Bajo tales circunstancias, tanto \trust como \familiarity influencian intenciones de comportamiento\cite{gefen2000commerce}.

\security influencia \trust. Por ejemplo, en el caso de \amazonNAME, una persona \familiarity con el concepto de \secureintcom podría permitir que se entreguen creencias específicas relativas a la medida de seguridad que ellos esperan del vendedor (\trust)\cite{gefen2000commerce}.

\ebay realizo un estudio en donde demostró que la reputación \ecommerceCOM de un \seller es un determinante, aunque no un determinante principal, del precio que los \sellers recibían en sus subastas. En ausencia de otros, recursos mas directos de información del \itemsCOM, un \consumer en \internetINT debe confiar en la precisión de la descripción del \itemsCOM del \sellers y la confiabilidad de la entrega del \itemsCOM del \sellers para decidir si, y cuanto esta dispuesto a ofertar por el bien. Dicho de otra manera, la reputación del \sellers se convierte en una consideración en la voluntad del \consumer para hacer una oferta en el artículo de la subasta. El resultado empírico muestra que un \seller con una mejor reputación puede esperar recibir ofertas superiores para sus \itemsCOM subastados. Sin embargo, aunque la reputación es una determinante estadísticamente significativa, su impacto tiende a ser pequeño\cite{melnik2002does}.

En este sentido, los nuevos \seller de subasta en el \websitesINT pueden considerar difícil competir con los \sellers existentes quienes tienen establecida una reputación positiva. Además, en la ausencia de un mecanismo que permitan que los \ratings sean transferidos entre sitios, un \seller quien ya tiene un alto \rating en un \websiteINT de subastas \online puede enfrentar un costo en cambiar a otro \websiteINT de subastas \online. \textit{Reputación} obviamente provee algunos beneficios para los \consumers buscando la \internetINT por \itemsCOM para los cuales realizar una oferta. Sin embargo, tal \textit{reputación} puede también implicar algunos costos a los \consumers, si el desarrollo de reputación para subastas \online actúa como barrera de entrada para nuevos \websitesINT de  subastas \online  y da poder de monopolio a los \websitesINT de subastas \online ya establecidos\cite{melnik2002does}.

%TODO
\subsubsection{Marketing Elements}

%TODO
\subsubsection{Analytics/reports}





Aunque no es relevante para el \frameworkPC, es importante destacar la importancia que tiene este tópico en el éxito que pueda tener una solución \ecommerceCOM. 

%TODO

\openSourcePC \ecommerceCOM \shoppingCarts ofrecen muchas ventajas para \textit{small businesses}. Soluciones \openSourcePC pueden ser desarrolladas para ajustarse a las necesidades del negociante. Ellos contienen una gran combinación de características a un mínimo costo. Y, aunque las opciones de soporte pueden ser mas limitadas que las propietarias o \textit{hosted platforms}, soluciones independientes \openSourcePC generalmente tienen grandes comunidades de desarrolladores y socios para ayudar a los nuevos comerciantes.