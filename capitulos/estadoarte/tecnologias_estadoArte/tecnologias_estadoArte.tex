
\section{Tecnologías }\label{cap:estadoArte:tecnologias}
%TODO agregar algo decente en este lugar.
A continuación se describen un conjunto de tecnologías que son agrupadas lógicamente de acuerdo a su función en el desarrollo de aplicaciones \webINT.

Estas tecnologías tienen en común que son herramientas que se ocupan en la actualidad además de tener la característica de ser \openSourcePC. 

\subsection{\dataBaseDB}

	\begin{itemize}
		\item
			\textbf{\mongodbNAME} (de "hu\textbf{mongo}us") es un base de datos \documentOriented \openSourcePC, escrita en \cPlusPlus \cite{technology_mongodb}. \mongodbNAME evita las estructuras de \dataBaseDB tradicionales \tableBasedDB en favor de \documentsDB \jsonLikeCPT con \schemasDB dinámicos (\mongodbNAME llama al formato \bsonNAME), haciendo la integración de \dataPC en cierto tipo de aplicaciones más fácil y rápido. \mongodbNAME es la \nosqlNAME líder, y ha alcanzando una popularidad que supera a \postgresql \cite{online_db_engines_ranking}.
			%MongoDB (from humongous) is a cross-platform document-oriented database. Classified as a NoSQL database, MongoDB eschews the traditional table-based relational database structure in favor of JSON-like documents with dynamic schemas (MongoDB calls the format BSON), making the integration of data in certain types of applications easier and faster. [http://en.wikipedia.org/wiki/MongoDB]
	
		\item
			\textbf{\mysqlNAME} es una \rdbms \openSourcePC que se posiciona como la  segundo más utilizado a nivel mundial \cite{online_db_engines_ranking}\cite{online_dispelling_myths}. \mysqlNAME es una opción popular de \dataBaseDB para aplicaciones \webINT ampliamente utilizado en el \stackAS \lampNAME (y otros \stacksAS, \ampNAME, etc.).
			%MySQL is a popular choice of database for use in web applications, and is a central component of the widely used LAMP open source web application software stack (and other 'AMP' stacks). LAMP is an acronym for "Linux, Apache, MySQL, Perl/PHP/Python." Free-software-open source projects that require a full-featured database management system often use MySQL.
	
		\item
			\textbf{\postgresql} es una \ordbms con énfasis en \extensibilityQA y normas de cumplimiento. Como \serverAS de \dataBaseDB, su primera función es \store \dataPC, de forma segura y apoyar las mejores prácticas, y recuperarlo más tarde, a petición de otras aplicaciones de \softwarePC, ya sea que estén \runningCPT en el mismo computador o \runningCPT en otro computador a través  de una \networkINT (incluido \internet). Puede manejar las cargas de trabajo que van desde pequeñas aplicaciones de una sola máquina a grandes aplicaciones orientados a \internet con muchos usuarios concurrentes.
			% (ORDBMS) with an emphasis on extensibility and standards-compliance. As a database server, its primary function is to store data, securely and supporting best practices, and retrieve it later, as requested by other software applications, be it those on the same computer or those running on another computer across a network (including the Internet). It can handle workloads ranging from small single-machine applications to large Internet-facing applications with many concurrent users. Recent versions also provide replication of the database itself for availability and scalability.
	
		\item
			\textbf{\sqlite} es una \rdbms contenida en una librería \cprogramming. A diferencia de otras \dbmangesystem, \sqlite no esta implementado como un proceso independiente del \clientAS. En contraste, es parte del programa en uso \cite{online_video_introduction_sqlite}.
			%is a relational database management system contained in a C programming library. In contrast to other database management systems, SQLite is not implemented as a separate process that a client program running in another process accesses. Rather, it is part of the using program.
	
		\item
			\textbf{\cassandraNAME} es un \nosqlNAME \wideColumnDB \store \openSourcePC diseñado para manejar gran cantidad de \dataPC sobre muchos \commodityServerPC, proporcionando \highAvailabilityDB sin \singlePointFailurePC. \cassandraNAME ofrece soporte robusto para \clustersAS que abarcan múltiples \dataCentersPC \cite{online_cassandra_multi_def}, con \masterlessDB \asynRepDB permitiendo \lowLatencyOperationsINT para todos los \clientsAS. \cassandraNAME es la \wideColumnDB \store más popular, y el segundo \nosqlNAME más utilizado\cite{online_db_engines_ranking}.
			%Apache Cassandra is an open source distributed database management system designed to handle large amounts of data across many commodity servers, providing high availability with no single point of failure. Cassandra offers robust support for clusters spanning multiple datacenters,[1] with asynchronous masterless replication allowing low latency operations for all clients.
	
		\item
			\textbf{\redisNAME} es un \nosqlNAME \openSourcePC, \keyValueDB \cachePC y \store. Usualmente referido como \serverAS con \dataPC estructurada dado que \keysDB pueden contener \stringsPL, \hashesPL, \listsPL, \setsPL, \sortedPL, \bitmapsPL y \hyperloglogsPL \cite{online_redis_website}.
			%Redis is an open source, BSD licensed, advanced key-value cache and store. It is often referred to as a data structure server since keys can contain strings, hashes, lists, sets, sorted sets, bitmaps and hyperloglogs.

		\item
			\textbf{\solrNAME} es \highly \reliableQA, \scalableQA y \faultTolerantQA, proporcionando \indexingDB distribuido, \replicationDB y \queryingDB \loadBalancedDB, \failoverPC y \recoveryDB automático, configuración centralizada y más \cite{online_official_website_solrn}. Sus principales característica corresponden a: soporte para búsqueda expresiones complejas, búsqueda de texto completa; \stemmingDB; \rankingCPT y agrupar los resultados de búsqueda; búsquedas \geoSpatialCPT; y búsqueda distribuida para \highScalabilityDB \cite{online_dbengines_solr_info}.
			%Solr is highly reliable, scalable and fault tolerant, providing distributed indexing, replication and load-balanced querying, automated failover and recovery, centralized configuration and more. 
		
\end{itemize}

\subsection{\serverSideAS}

	\subsubsection{\webserverINT \softwarePC }
		\begin{itemize}
			\item 
				\textbf{\nodejsNAME} es una plataforma construida sobre \chrome's \javaScriptNAME \runtimeCPT para construir fácilmente aplicaciones \networkINT rápidas y escalables. \nodejsNAME utiliza un modelo \eventdrivenPL, \nonbloking  \inputOutput que hace esto liviano y eficiente, muy adecuado para aplicaciones \dataintensive \realTimeINT que corren sobre dispositivos distribuidos \cite{technology_nodejs}.
				%is a platform built on Chrome's JavaScript runtime for easily building fast, scalable network applications. Node.js uses an event-driven, non-blocking I/O model that makes it lightweight and efficient, perfect for data-intensive real-time applications that run across distributed devices \cite{technology_nodejs}.
			
			\item
				\textbf{\apacheNAME \httpNAME \serverAS }, coloquialmente llamado \apacheNAME, es el \softwarePC del \serverAS \webINT mas utilizado en el mundo \cite{online_technology_apache}.
				 %The Apache HTTP Server, colloquially called Apache, is the world's most widely-used web server software \cite{online_technology_apache}.
			
			\item
				\textbf{\apacheNAME \tomcatNAME }	es un \webserverINT y contenedor de \servletNAME  \openSourcePC. \tomcatNAME implementa muchas de las especificaciones de \javaeeNAME incluyendo \javaNAME \servletNAME, \javaspNAME, \javaelNAME, y \websocketINT, y proveer un ambiente \webserverINT \httpNAME para que código \javaNAME \runInCPT \cite{online_technology_officialsite_tomcat}.
				%Apache Tomcat is an open-source web server and servlet container developed by the Apache Software Foundation (ASF). Tomcat implements several Java EE specifications including Java Servlet, JavaServer Pages (JSP), Java EL, and WebSocket, and provides a "pure Java" HTTP web server environment for Java code to run in.
		\end{itemize}

	\subsubsection{\serverSideAS \scriptingLanguagePL}
		\begin{itemize}
			\item
				\textbf{\phpNAME} es un lenguaje \scripting \serverSideAS diseñado para desarrollo \webINT pero también utilizado como un lenguaje de programación de uso general\cite{online_technology_php}. \phpNAME es soportado por varios \webserverINT incluyendo \apacheNAME \httpNAME \serverAS.
				%is a server-side scripting language designed for web development but also used as a general-purpose programming language. PHP has a direct module interface called Server Application Programming Interface (SAPI), which is supported by many web servers including Apache HTTP Server

			\item
				\textbf{\javaScriptNAME} es un lenguaje \lightweightPL, interpretado, \objectOrientedPL con \firstClassFuncPL, mayoritariamente conocido como \scriptingLanguagePL para páginas \webINT, pero utilizado en muchos ambientes \nonBrowserINT también tales como \nodejsNAME y \apacheNAME \couchdbNAME. Es un \scriptingLanguagePL \prototypeBasedPL, \multiParadigmPL dinámico, y soporta estilos de programación \objectOrientedPL, \imperativePL, y \functionalPL \cite{online_technology_mozilla_javascript}. 
				%JavaScript® (often shortened to JS) is a lightweight, interpreted, object-oriented language with first-class functions, most known as the scripting language for Web pages, but used in many non-browser environments as well such as node.js or Apache CouchDB. It is a prototype-based, multi-paradigm scripting language that is dynamic, and supports object-oriented, imperative, and functional programming styles.
			\item
				\textbf{\javaNAME} es un lenguaje de programación de propósito general que es \objectOrientedPL, \concurrentPL \classBasedPL \cite{online_technology_mozilla_javascript}. \javaNAME tiene la filosofía de \writeOnceRunAnyPL, lo que significa que si se quiere cambiar de plataforma, no es necesario volver a compilar el código \cite{online_java_write_once}.
				%Java is a general-purpose computer programming language that is concurrent, class-based, object-oriented,[11] and specifically designed to have as few implementation dependencies as possible. It is intended to let application developers "write once, run anywhere" (WORA),[12] meaning that code that runs on one platform does not need to be recompiled to run on another.[13]

			\item
				\textbf{\rubyNAME} es un lenguaje de programación dinámico y \openSourcePC, con énfasis en \simplicity y \productivity. Tiene un sintaxis elegante que es natural para leer y fácil para escribir. \rubyNAME es un lenguaje de programación \objectOrientedPL, \imperativePL, y \functionalPL \cite{online_ruby_org}.
				%A dynamic, open source programming language with a focus on simplicity and productivity. It has an elegant syntax that is natural to read and easy to write.

			\item
				\textbf{\perlNAME} es un lenguaje de programación con más de 27 años de desarrollo. \perlNAME 5 corre en más de 100 plataformas desde \portablesAS a \mainframesAS, y es adecuado tanto para un rápido \prototypingCPT y desarrollar proyectos a gran escala \cite{online_org_perl_about}.
				%Perl 5 is a highly capable, feature-rich programming language with over 27 years of development. Perl 5 runs on over 100 platforms from portables to mainframes and is suitable for both rapid prototyping and large scale development projects.

			%TODO add cite documentation and fix paradigmas
			\item
				\textbf{\pythonNAME} es un lenguaje de programación \highLevelCPT de propósito general, permitiendo desarrollar conceptos en unas pocas lineas, lo cual sería imposible en lenguajes como \javaNAME ó \cPlusPlus. \pythonNAME es un lenguaje de programación \multiParadigmPL dinámico, soportando paradigmas de programación \objectOrientedPL, \imperativePL, \functionalPL, \proceduralPL y \reflectivePL \cite{online_org_docs_python_functional}.
		\end{itemize}

	\subsubsection{\frameworksPC}
		\begin{itemize}

			\item \textbf{\cakephpNAME}
			%TODO
			\item
				\textbf{\laravelNAME} es un \frameworkPC de aplicaciones \webINT \phpNAME \freePC y \openSourcePC, designado para el desarrollo de aplicaciones \webINT utilizando el modelo de arquitectura \mvcAS. De acuerdo a una encuesta realizada a desarrolladores de \phpNAME, \laravelNAME fue elegido como el \frameworkPC \phpNAME mas popular del 2013 \cite{online_sitepoint_best_php_frameworks_2014}. En la actualidad ( Febrero 2015 ), \laravelNAME es el proyecto \phpNAME en \gitHubNAME más popular \cite{online_popularity_php_proyects}.
				%Laravel is a free, open source PHP web application framework, designed for the development of model–view–controller (MVC) web applications. Laravel is released under the MIT License, with its source code hosted on GitHub.[3][4][5]
				%According to a December 2013 developers survey on PHP frameworks popularity, Laravel was listed as the most popular PHP framework of 2013, followed by Phalcon, Symfony2, CodeIgniter and others.[6] As of August 2014, Laravel is the most popular and watched PHP project on GitHub.[7]
			\item
				\textbf{\phalconNAME} es un \frameworkPC \webINT para \phpNAME \highPerformanceQA basado en el patrón arquitectónico \mvcAS. Es un \frameworkPC \openSourcePC implementado como una extención de \cNAME para optimizar \performanceQA y bajo consumo de \resourcesCPT \cite{online_technology_officialsite_phalcon}. \phalconNAME ha sido considerado como el segundo \frameworkPC \phpNAME más popular \cite{online_popularity_php_proyects}.
				%Phalcon is a web framework implemented as a C extension offering high performance and lower resource consumption

				%Phalcon is a high-performance web framework for PHP based on the MVC pattern. Originally released in 2012, it is an open-source framework issued under the BSD license. Unlike most PHP frameworks, Phalcon is implemented as an extension written in C/C++ in order to optimize performance. This is intended to boost execution speed and resource usage with the goal of handling more requests per second than comparable frameworks written primarily in PHP. One drawback of this approach is that root/administrative access is required on the server in order to install Phalcon (via building or pre-compiled binary).
			\item \textbf{\symfonyNAME} es un conjunto de componentes \phpNAME, un \frameworkPC de aplicaciones \webINT, y una comunidad - todo trabajando en armonía \cite{online_technology_officialsite_symfony}. \symfonyNAME es un \softwarePC \freePC para desarrollar aplicaciones con el patrón arquitectónico \mvcAS. Es el tercer \frameworkPC más popular de \phpNAME \cite{online_popularity_php_proyects}.
			%« Symfony is a set of PHP Components, a Web Application framework, a Philosophy, and a Community — all working together in harmony. »
			%Symfony is a PHP web application framework for MVC applications. Symfony is free software and released under the MIT license. The symfony-project.com website launched on October 18, 2005.[2]
			\item \textbf{\drupalNAME}
			%TODO
			\item \textbf{\zendNAME}
			%TODO
			\item \textbf{\yiiNAME}
			%TODO
			\item \textbf{\wordPressNAME}
			%TODO

			\item
				\textbf{\rubyonrailsNAME} es un \frameworkPC para el desarrollo de aplicaciones \webINT escrito en lenguaje \rubyNAME. Esta diseñado para programar fácilmente aplicaciones \webINT haciendo supuestos sobre que necesita cada desarrollador para iniciar. Esto permite escribir menos código que muchos otros lenguajes y \frameworksPC \cite{online_technology_rubyonrails}.
				%Rails is a web application development framework written in the Ruby language. It is designed to make programming web applications easier by making assumptions about what every developer needs to get started. It allows you to write less code while accomplishing more than many other languages and frameworks.

			\item
				\textbf{\sinatraNAME} es un \domainSpecificLangPL para construir \websitesINT, \webServiceINT y aplicaciones \webINT en \rubyNAME. Enfatiza un enfoque para desarrollo \minimalisticQA, ofreciendo solo lo esencial para manejar \httpNAME \requestINT y entregar \responsesINT a los \clientsAS. \sinatraNAME no fuerza a seguir el patrón \modelViewConAS, u otro patrón que importe. Es un \wrapperAS \lightweightPL alrededor de \rackMiddleRubyAS y fomenta una relación cercana entre \serviceCPT \apiendpointsAS y los \httpVerbsAS, transformando esto particularmente ideal para \webServicesINT y \apisAS \cite{book_sinatra_oreilly}.

				\sinatraNAME no es un \frameworkPC; no hay herramientas \builtINPL \ormAS, no hay archivos de configuración \preFabCPT; incluso no hay una carpeta de proyecto a menos que sea creada. Ciertamente esto puede parecer con algo no deseable, pero sí es algo completamente liberador. Las aplicaciones en \sinatraNAME son muy flexibles por naturaleza, típicamente no son más grandes de lo que necesitan para ser y pueden ser fácilmente distribuidas como \rubyGemsAS  \cite{book_sinatra_oreilly}.
				%Sinatra is a domain-specific language for building websites, web services, and web applications in Ruby. It emphasizes a minimalistic approach to development, offering only what is essential to handle HTTP requests and deliver responses to clients.
				%Sinatra does not force you to adhere to the model-view-controller pattern, or any other pattern for that matter. It is a lightweight wrapper around Rack middleware and encourages a close relationship between service endpoints and the HTTP verbs, making it particularly ideal for web services and APIs (application programming interfaces).

				%Sinatra is not a framework; you’ll find no built-in ORM (object-relational mapper) tools, no pre-fab configuration files...you won’t even get a project folder unless you create one yourself.

				%While it may not seem like it now, this can be quite liberating. Sinatra applications are very flexible by nature, typically no larger than they need to be and can easily be distributed as gems.

			\item
				\textbf{\padrinoNAME} es un \frameworkPC construido sobre la librería \webINT \sinatraNAME.  \padrinoNAME fue creado para realizar de forma sencilla y entretenida, código para aplicaciones \webINT avanzadas mientras mantiene el espíritu que hace a \sinatraNAME excelente \cite{online_technology_padrino_officialsite}.
				%Padrino is a ruby framework built upon the Sinatra web library. Sinatra is a DSL for creating simple web applications in Ruby. Padrino was created to make it fun and easy to code more advanced web applications while still adhering to the spirit that makes Sinatra great!

			%\item \textbf{\nynyNAME}.
			 %TODO
			
			\item
				\textbf{\djangoNAME} es una \frameworkPC \openSourcePC escrito en \pythonNAME, diseñado para perder \couplingAS y obligar la separación entre diferentes partes de la aplicación \cite{online_book_django_developmentpattern}.


			\item
				\textbf{\expressjsNAME} es un \frameworkPC de aplicaciones \webINT \nodejsNAME minimalistas y flexibles, proporcionando un conjunto robusto de características  para crear aplicaciones \webINT \single, \multipage, e híbridas \cite{online_technology_expressjs_officialsite}.
				%Express is a minimal and flexible node.js web application framework, providing a robust set of features for building single and multi-page, and hybrid web applications.

			\item
				\textbf{\koaNAME} Es un \frameworkPC \webINT diseñado por el equipo detrás de \expressjsNAME, cuyo objetivo es ser la base más pequeñas, más \expressiveQA, y más \robustQA para las aplicaciones \webINT y \apisAS. El generador de \koaNAME permite deshacerse de \callbacksPL e incrementar considerablemente el \errorHandlingPL. \koaNAME no incorpora ningún \middlewareAS en su \coreAS, y provee una elegante \suitePC de métodos que hacen la escritura en el \serverAS \fastQA y \enjoyableQA \cite{online_technology_koa_officialsite}.
				%Koa is a new web framework designed by the team behind Express, which aims to be a smaller, more expressive, and more robust foundation for web applications and APIs. Through leveraging generators Koa allows you to ditch callbacks and greatly increase error-handling. Koa does not bundle any middleware within core, and provides an elegant suite of methods that make writing servers fast and enjoyable.
		\end{itemize}

\subsection{\clientSideAS}
	\begin{itemize}
		\item 
			\textbf{\angularjsNAME} es un \frameworkPC \clientSideAS desarrollado por \googleNAME. Esta escrito en \javaScriptNAME, con una librería reducida de \jqueryNAME. La teoría detrás de \angularjsNAME es proveer de un \frameworkPC que hace sencilla la implementación de aplicaciones y \webPageINT \wellDesignQA y \structuredQA, usando un \frameworkPC \mvcAS \cite{book_technology_node_mongo_angular_development}. 

			\angularjsNAME es utilizado para abordar muchos de los problemas encontrados en el desarrollo de \singlePageAppINT \cite{technology_angularjs}.

		\item 
			\textbf{\emberjsNAME} \cite{online_technology_emberjs} es un \frameworkPC \openSourcePC \clientSideAS \javaScriptNAME para aplicaciones \webINT basado en el patrón de arquitectura \mvcAS. Esto permite a los desarrolladores crear \singlePageAppINT \scalablesQA \cite{online_Enterprise_Moving_SinglePage_Design}, con la incorporación de expresiones idiomáticas comunes y las mejores prácticas dentro de un  \frameworkPC que provee un enriquecido \objectModelPL, \dataBindingPL \declarativePL \twoWayINT, \computedPropEmberAS, \templatesAS \autoUpdatedAS \poweredCPT por \handlebarsNAME y un \routerPC para manejar los estados de la aplicación \cite{online_technology_emberjs_getting_into}.
			%Ember.js is an open-source client-side JavaScript web application framework based on the model-view-controller (MVC) software architectural pattern. It allows developers to create scalable single-page applications[1] by incorporating common idioms and best practices into a framework that provides a rich object model, declarative two-way data binding, computed properties, automatically-updating templates powered by Handlebars.js, and a router for managing application state.[2]

		\item
			\textbf{\backbonejsNAME} da estructura a aplicaciones \webINT ofreciendo \textbf{\modelsCustom} con asociación \keyValue y \events \custom, \textbf{\collectionsDB} con una \api enriquecida de funciones, \textbf{\viewsAS} \handling \event declarativos, y conecta todo a su \api existente sobre una interfaz \jsonNAME \restful \cite{online_technology_backbone}.
			%gives structure to web applications by providing models with key-value binding and custom events, collections with a rich API of enumerable functions, views with declarative event handling, and connects it all to your existing API over a RESTful JSON interface.
		

		\item
			\textbf{\phonegapNAME} es un \frameworkPC \freePC y \openSourcePC que permite crear \appsINT \mobileINT usando \webINT \apisAS para las plataformas que interesan, en lugar de confiar en una \apisAS \platformSpecificCPT como aquellas de \iosNAME, \windowsPhoneNAME ó \androidNAME \cite{online_technology_phonegap_mobile_app_plataforms}.
			%PhoneGap is a free and open source framework that allows you to create mobile apps using standardized web APIs for the platforms you care about.
			%instead of relying on platform-specific APIs like those in iOS, Windows Phone, or Android.[5] It enables wrapping up of HTML, CSS and Javascript code depending upon the platform of the device. It extends the features of HTML and Javascript to work with the device. The resulting applications are hybrid, meaning that they are neither truly native mobile application (because all layout rendering is done via web views instead of the platform's native UI framework) nor purely web-based (because they are not just web apps, but are packaged as apps for distribution and have access to native device APIs). Mixing native and hybrid code snippets has been possible since version 1.9.

		\item \textbf{\extjsNAME}

		\item
			\textbf{\jqueryNAME} es una librería \javaScriptNAME \fastQA, \smallQA y \featureRichQA. Esto hace que cosas como la manipulación y recorrido del \htmlDocumentINT, \eventHandlingPL, animación y \ajaxNAME mucho más simple con una \apisAS fácil de utilizar que funciona a través de multitud de \browsersINT. Con una combinación de \versatilityQA y \extensibilityQA, \jqueryNAME ha cambiado la manera que millones de personas escriben \javaScriptNAME \cite{online_technology_jquery_officialSite}.
			%jQuery is a fast, small, and feature-rich JavaScript library. It makes things like HTML document traversal and manipulation, event handling, animation, and Ajax much simpler with an easy-to-use API that works across a multitude of browsers. With a combination of versatility and extensibility, jQuery has changed the way that millions of people write JavaScript.
		
		\item \textbf{\bootstrap} es el \frameworkPC más popular de \htmlNAME, \cssNAME, y \javaScriptNAME para el desarrollo del primer proyecto \mobileINT crítico en la \webINT \cite{technology_bootstrap}.
		%is the most popular HTML, CSS, and JS framework for developing responsive, mobile first projects on the web \cite{technology_bootstrap}.
	\end{itemize}

\subsection{\fullstackAS \javaScriptNAME \frameworksPC}
\label{cap:estadoArte:section:fullstack_javaScript_framework}

\subsubsection{\stackAS de \javaScriptNAME \frameworksPC}
	\begin{itemize}
		\item
			\textbf{\meanstackNAME} (\mongodbNAME, \expressjsNAME, \angularjsNAME, \nodejsNAME) es un \fullstackAS \javaScriptNAME \frameworkPC que simplifica y acelera el desarrollo de aplicaciones \webINT \cite{online_mean_io}. Existen varias variantes al \stackAS \meanstackNAME, tales como \meenstackNAME. En ocasiones el términos es utilizado simplemente para indicar infraestructura corriendo en \nodejsNAME en combinación con una \dataBaseDB \nosqlNAME.
		\item
			\textbf{\meenstackNAME} \cite{online_meen_github}(\mongodbNAME, \emberjsNAME, \expressjsNAME, \nodejsNAME) es un \fullstackAS \javaScriptNAME \frameworkPC equivalente a \meanstackNAMEref.
		\item
			\textbf{\socketStreamNAME}\cite{online_socketstream_official_org}
		\item
			\textbf{\saneNAME} es un \fullstackAS \javaScriptNAME que permite rápidamente crear aplicaciones \webINT \productionReadyPC utilizando \sailsNAME y \emberjsNAME. Tiene soporte \dockerNAME, \generatorsAS y más \cite{online_sanestack_official_site}.
			%A Javascript Fullstack and CLI that lets you rapidly create production-ready web apps using Sails and Ember. Get Docker support, generators and more.
		\item
			\textbf{\cokeNAME} es un \nodejsNAME \mvcAS \frameworkPC \lightweightPL que \speedupCPT el desarrollo \webINT. Es \modularizedAS \cite{online_cokejs_official_site}. 
			%COKE is a lightweight node.js MVC framework that speeds up your web development. It's simple, it's modularized, it's somking fast!”
		\item
			\textbf{\sleekjsNAME} es un \frameworkPC \mvcAS implementado de \nodejsNAME, \builtINPL con dependencias base \handlebarsNAMEref, \expressjsNAMEref, \mongooseNAME. La arquitectura de \sleekjsNAME sigue el formato común de \mvcAS lo que hace sencillo manejar y construir nuevos \websitesINT\cite{online_sleekjs_official_site}.
			%is an MVC Framework implemented from Node.js, built-in with base dependency on handlebars.js, express.js, mongoose. Sleek.js architecture follows common format of MVC which makes it easy to handle and build better sites.
		\item
			\textbf{\danfNAME} es un \fullstackAS \javaScriptNAME/\nodejsNAME \oopPL \frameworkPC permitiendo escribir código de la misma manera tanto en el \serverSideAS y \clientSideAS. Provee muchas caracteristicas produciendo aplicaciones \scalableQA, \maintainableQA, \testeableQA y \performanceQA\cite{online_danf_official_gitHub}.
			%Danf is a javascript/node.js full-stack OOP framework allowing to code the same way on both the server (node.js) and client (browser) sides. It provides many features in order to help producing scalable, maintainable, testable and performant applications.

	\end{itemize}
\subsubsection{\isomorphicJSFwASref}
	\begin{itemize}
		\item
			\textbf{\meteorNAME} es una \frameworkPC  \openSourcePC \realTimeINT para aplicaciones \webINT \javaScriptNAME escritas sobre \nodejsNAME \cite{online_meteor_documentation}. \meteorNAME permite \prototypingCPT muy veloz \cite{online_meteor_documentation_why} y produce código \crossPlatform (\webINT, \androidNAME, \iosNAME) \cite{online_meteor_cross_platform}.

		\item
			\textbf{\derbyNAME} \cite{online_technology_derby_officialsite} es un \frameworkPC \fullstackAS, \isomorphicAS, con patrón arquitectónico \mvcAS que hace sencillo escribir aplicaciones \collaborativeQA, \realTimeINT que \runCPT tanto en \nodejsNAME como en \browsersINT \cite{online_technology_isomorphic_javascript_frameworks}.
			%Full-Stack, Isomorphic, MVC framework that makes it easy to write realtime, collaborative applications that run in both Node.js and browsers
		
		\item
			\textbf{\mojitoNAME} es un \frameworkPC para aplicaciones \mvcAS construido sobre \yuiThreeNAME que permite el desarrollo ágil de aplicaciones \webINT. \mojitoNAME permite a los desarrolladores usar una combinación de configuraciones y una arquitectura \mvcAS para crear aplicaciones. \mojitoNAME fue diseñado para correr tanto en el \serverSideAS (\nodejsNAME) como en el \clientAS (\browserINT) \cite{online_technology_mojito_officialsite_docs_intro}
			%Mojito is a model-view-controller (MVC) application framework built on YUI 3 that enables agile development of Web applications. Mojito allows developers to use a combination of configuration and an MVC architecture to create applications. Because client and server components are both written in JavaScript, Mojito can run on the client (browser) or the server (Node.js).

		\item
			\textbf{\rendrNAME} es una pequeña librería de \airbnbNAME \cite{online_technology_airbnb_officialsite} que permite \runCPT aplicaciones \backbonejsNAME sin problemas tanto en \clientSideAS como en \serverSideAS. Permite al \webserverINT servir páginas \htmlNAME \fullyFormedCPT a cualquier profundidad de \linkINT de la aplicación, preservando al mismo tiempo la sensación ágil de una aplicación \mvcAS \clientSideAS \backbonejsNAME tradicional \cite{online_technology_isomorphic_javascript_frameworks}.
			%Rendr is a small library from Airbnb that allows you to run your Backbone.js apps seamlessly on both the client and the server. Allow your web server to serve fully-formed HTML pages to any deep link of your app, while preserving the snappy feel of a traditional Backbone.js client-side MVC app..
		\item
			\textbf{\flatironNAME} es un \frameworkPC adaptable para construir aplicaciones \webINT modernas. Esto fue construido desde cero para ser utilizado con \javaScriptNAME y \nodejsNAME \cite{online_technology_flatiron_officialsite}.
			%Flatiron is an adaptable framework for building modern web applications. It was built from the ground up for use with JavaScript and Node.js. 

	\end{itemize}

	
%%%%%%%%%%%%%%%%%%%%%%%%%%%%%%%%%%%%%%%%%%%%%%%%%%%%%%%%%%%%%%%%%%%%%%%%%%%%%
%%%%%%%%%%%%%%%		    RESUME FULL STACK JAVASCRIPT		  %%%%%%%%%%%%%%%
%%%%%%%%%%%%%%%%%%%%%%%%%%%%%%%%%%%%%%%%%%%%%%%%%%%%%%%%%%%%%%%%%%%%%%%%%%%%%

\begin{table}[H]
    \centering
%\begin{tabular}{ |C{0.3\paperwidth}|C{0.3\paperwidth}| }
\begin{tabular}{ |l|c| }

\hline
	&
	Tipo \frameworkPC%&


\\ \hline
	\derbyNAMEref &
	\isomorphicJSFwASref%&
	%\frameworkPC
	
\\ \hline
	\socketStreamNAMEref &
	\socketStreamNAMEref, \emberjsNAME, \nodejsNAME%&
	%\stackAS
	
\\ \hline
	\meanstackNAMEref &
	\mongodbNAME, \expressjsNAME, \angularjsNAME, \nodejsNAME.%&
	%\stackAS
	
\\ \hline
	\meenstackNAMEref &
	\mongodbNAME, \emberjsNAME, \expressjsNAME, \nodejsNAME.%&
	%\stackAS
 
\\ \hline
	\meteorNAMEref &
	\isomorphicJSFwASref%&
	%\frameworkPC
	
\\ \hline
	\mojitoNAMEref &
	\isomorphicJSFwASref%&
	
\\ \hline
	\saneNAMEref &
	\sailsNAME, \emberjsNAME, \nodejsNAME. %&
	%\stackAS
	
\\ \hline
	\cokeNAMEref &
	%&
	
\\ \hline
	\sleekjsNAMEref &
	\handlebarsNAME, \expressjsNAME, \mongooseNAME, \nodejsNAME.%&
	%\stackAS
	
\\ \hline
	\danfNAMEref &
	%&

\\ \hline
\end{tabular}
    \caption{Resúmen \fullStackJSFwASref}
    \label{tab:resume_full_stack_javaScript}
\end{table}


\subsection{\toolsCPT para el desarrollo.}

	\begin{itemize}
		\item
			\textbf{\grunttoolNAME} es una herramienta \commandLine \taskBased construida para proyectos \javaScriptNAME, que corren sobre \nodejsNAME y es instalado vía \npm \cite{technology_gruntjs}
			%Grunt is a task-based command-line build tool for JavaScript projects, that runs on Node.js and is installed via npm.
			%TOOD  
			%Task runner for automating build and development worklow

		\item
			\textbf{\browserifyNAME} permite \requireINT \modulesAS en el \browsersINT \bundlingUpCPT todas las dependencias \cite{online_official_website_browserify}. \browsersINT no necesitan definir el método \requestINT, pero \nodejsNAME si. Con \browserifyNAME es posible \writeCPT código que use \requireINT de la misma manera que se usaría en \nodejsNAME  \cite{online_official_website_browserify}.
			%lets you require modules in the browser by bundling up all of your dependencies.
			%Browsers don't have the require method defined, but Node.js does. With Browserify you can write code that uses require in the same way that you would use it in Node.

		\item
			\textbf{\dockerNAME} es una plataforma \openSourcePC y \sysadmins para construir, enviar y ejecutar aplicaciones distribuidas. Consiste en \dockerNAME \engine, un portable, \lightweightPL \runtimeCPT y \packagingCPT \tool, y \dockerNAME \hub, un servicio de la nube para compartir aplicaciones y \workflowsCPT automatizados, \dockerNAME permite que las aplicaciones sean rápidamente ensamblados desde componentes y elimina la fricción entre \developmentPC, \qaSIGLA y \productionPC \environmentsPL. Como resultado, puede ser entregado rápidamente y correr la misma aplicación, sin cambios, en \laptops, \dataPC \centerCustom \vmsSIGLA, y cualquier otro recurso provisto por \internet \cite{technology_docker}.
			%is an open platform for developers and sysadmins to build, ship, and run distributed applications. Consisting of Docker Engine, a portable, lightweight runtime and packaging tool, and Docker Hub, a cloud service for sharing applications and automating workflows, Docker enables apps to be quickly assembled from components and eliminates the friction between development, QA, and production environments. As a result, IT can ship faster and run the same app, unchanged, on laptops, data center VMs, and any cloud\cite{technology_docker}.

		\item
			\textbf{\coffeescript} es un lenguaje pequeño que se compila dentro de \javaScriptNAME. Debajo de ese poco adecuado lenguaje, \javaScriptNAME tiene un hermoso corazón. \coffeescript es un intento de exponer las partes buenas de \javaScriptNAME de una forma sencilla \cite{technology_coffeescript}.
		
			%is a little language that compiles into JavaScript. Underneath that awkward Java-esque patina, JavaScript has always had a gorgeous heart. CoffeeScript is an attempt to expose the good parts of JavaScript in a simple way\cite{technology_coffeescript}.
			
			%The golden rule of \coffeescript is: "It's just JavaScript". The code compiles one-to-one into the equivalent JS, and there is no interpretation at runtime. You can use any existing JavaScript library seamlessly from CoffeeScript (and vice-versa). The compiled output is readable and pretty-printed, will work in every JavaScript runtime, and tends to run as fast or faster than the equivalent handwritten JavaScript \cite{technology_coffeescript}.
	\end{itemize}