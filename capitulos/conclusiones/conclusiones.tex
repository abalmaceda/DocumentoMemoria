%!TEX root = ../memoria.tex

%%%%%%%%%%%%%%%%%%%%%%%%%%%%%%%%%%%%%%%%%%%%%%%%%%%%%%%%%%%%%%%%%%%%%%%%%%%%%
%%%%%%%%%%%%%%%%%%%%%%%%%%%%%    Conclusiones 	%%%%%%%%%%%%%%%%%%%%%%%%%%%%%
%%%%%%%%%%%%%%%%%%%%%%%%%%%%%%%%%%%%%%%%%%%%%%%%%%%%%%%%%%%%%%%%%%%%%%%%%%%%%

%TODO: falta
\chapter{Conclusiones} \label{cap:conclusiones}
	
	La plataforma desarrollada no constituye una solución completa de un \frameworkPC \ecommerceCOM, pero si una sólida base para esta finalidad dado que cada una de sus componentes fueron seleccionadas a partir de las mejores prácticas que se han identificado producto de la experiencia en el mercado y diversos estudios realizados.

	La tecnología utilizada mostró ser una elección correcta dado las facilidades que esta entregó, junto con la gran comunidad aportando soluciones, permitieron enfocarse principalmete en el problema de negocio en lugar de enfocar los esfuerzos para el uso de la tecnología. Sin duda esta característica es altamente relevante en donde la velocidad de desarrollo presenta una gran ventaja competiviva para adaptarse rápidamente a las futuras necesidades del mercado.

	Para desarrollar una aplicación de grandes proporciones y que además permitiriera llevó a la construcción de una arquitectura genérica la cual puede ser utilizada como punto de inicio de una gran cantidad de aplicaciones \webINT.  
	Aunque no fue planteado como un objetivo inicial, uno de los lógros más significativos fue el desarrollo de esta arquitectura, la cual es un punto sólido de inicio para practicamente cualquier aplicación \webINT de la actualidad.

	De los objetivos planteados inicialmente, no se cumplío el referente a la evaluación del uso del \frameworkPC para el desarrollo de una solución particular de servicio \ecommerceCOM.
	El respecto de los demás objetivos planteados al comienzo, sí se cumplieron puesto que se logró desarrollar un \frameworkPC que si bien es cierto, no constituye una solución equivalente en funcionalidades a los \frameworkPC descritos en un inicio dada la gran catidad de características que estos tienen, si representa el \coreAS de una plataforma con carácterísticas nuevas, útiles para el desarrollo de plataformas modernas.

	% \meteorNAME es una plataforma flexible


	% durante el proceso de desarrollo de la plataforma pude confirmar la gran cantidad de funcionalidades creadas y empaquetadas por parte de la comunidadd, lo cual permitio enfocar mis esfuerzos en lo que queria lograr con la aplicación.
	% Ademas esta cualidad de los packages, junto con las herramientasa que entrega meteor, permitieron desarrollar una arquitectura 





	% Conclusion
	% Even if you implement every one of the above suggestions and see noticeable improvements with regards to your abandonment rate and conversion rate, don't rest on your laurels. Like SEO and web design in general, effective shopping cart design is an ongoing process. What works well now may fizzle out later. Without keeping track of statistics and important metrics, you could end up losing out a lot down the road. When you see a dip in sales or conversions or an increase in your abandoned shopping cart rate, you'll be able to quickly implement changes that will hopefully get things back on an even keel again. This proactive approach will serve you well and should help you achieve unparalleled success with your e-commerce site.
	% http://www.imediaconnection.com/content/36794.asp





	% Success of e-Commerce websites lies in improving user experience, keeping it simple, and winning client’s trust. This will not only result in converting potential clicks into final transaction payment, but also strongly influence the customer to revisit your website in future.