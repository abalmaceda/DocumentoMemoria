%!TEX root = ../../memoria.tex
\chapter{Caracteristicas faltantes}

	\section{Estabilidad del sistema}
		Básicamente solucionar bugs que hacen que el sistema no se comporte adecuadamente.

	\section{Templates}
		Existen ciertos elementos de las interfaces que aun no estan integradas 100\%  con los Templates. Esto implica que al cambiar de template, el color del elemento no cambia.
		Es un detalle menor en términos de implementación, pero relevante a la hora de diseñar un sitio \ecommerceCOM que se adeque a las necesidades.

	\section{Variantes de productos}
		El sistema debe permitir agregar variantes de productos. Un buen ejemplo de esto es cuando se venden zapatillas que son iguales salvo por la talla. No tendría sentido agregar un nuevo producto completo, solo para poder vender el mismo producto con una talla diferente.

	\section{Ordenes}
		Falta el manejo de las ordenes de compra. Las interfaces estan listas, solo falta trabajar en el backend.

	\section{Proceso Shipping completo}
		Actualmente es posible ver la vista de shipping y sus diferentes elementos sin nigún tipo de inconveniente. Sin embargo el sistema aun carece de \textit{guiar al usuario}. Por ejemplo el estado global del sistema no se actualiza automáticamente aun.
		Básicamente lo que falta acá es terminar con la máquina de estados que actualmente define este proceso.

	\section{Sistema de Pago}
		El sistema no tiene configurado un método para agregar sistema de pago. La idea tras esto es crear un package que permita administrar sistemas de pago. Cada nuevo sistema de pago será visto como un nuevo package.
		Como mínimo, se implementará el administrador de sistemas de pago junto con un sistema de pago como Pay Pal, Google Wallet, etc.

	\section{Agregar archivos de configuración}
		Agregar ciertos archivos que permitan hacer configuraciones más sencillas del sistema sin tener la necesidad de involucrarse directamente con el código.
		Por ejemplo la máquina de estados del proceso shipping debe estar descrito utilizando un archivo.

	\section{Analytics}

		Agregar alguna herramienta para realizar analytics, tal como Google Analytics.
		Estas herramientas son de vital importancia para hacer marketing inteligente y desiciones de negocio.

	\section{Logging out de otras secciones}
		%Today, it's important to provide users with the ability to log out of not only their current session, but also any sessions they may have open from other devices. Otherwise what is a user to do if their mobile phone is lost or stolen, with a live login session still on the device? With Meteor Accounts this is as simple as providing a button that when clicked calls Meteor.logoutOtherClients.
		En la actualidad es importante permitir a los usuarios no solo cerrar la sesión actual, si no tambien la de todos los dispositivos que tengan una sesión abierta. De esta manera solucionar el típico problema de un teléfono robado o perdido con una sesión activa.





