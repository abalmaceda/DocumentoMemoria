%!TEX root = ../../memoria.tex
\chapter{Trabajos Futuros}\label{cap:trabajos_futuros}

	% Mejoras en la arquitectura
	%!TEX root = ../../../memoria.tex

\section{Mejora en la arquitectura genérica}
	%TODO Falta
	Durante el desarrollo de la aplicación se comprendió el fuerte concepto que representa \packageAS para \meteorNAME. Esto sugirió mejoras en la arquitectura las cuales básicamente corresponden en desarrollar un proyecto base que administre todos los módulos (\packageAS) que se agregan al sistema para incorporar caracteristicas a la aplicación en desarrollo. 

	Cada uno de estos módulos, contendría un archivo descriptivo que expondría:
	\begin{itemize}
		\item
			\textbf{Permisos}. Se describen los permisos necesarios que el usuario necesita para utilizar los recursos del módulo. Se utiliza el \packageAS \alanningRolesPackage para soportar esta funcionalidad.
		\item
			\textbf{Enrutamientos}. Los enrutamientos para acceder a los servicios del módulo.
		\item
			\textbf{Configuración}. El enrutamiento para acceder al Menú. Este debería ser accesible desde el \dashboardEF  de la arquitectura base.
		\item
			\textbf{Tipo de Módulo}. Agregar etiquetas que permitan distinguir y agrupar diferentes módulos.
		\item
			\textbf{Etiqueta descriptiva}. Básicamente un nombre con el cual podría distinquir el módulo. Por ejemplo podria utilizarse en la interfaz de \dashboardEF para acceder a su menú de configuración.
		\item
			\textbf{Icono}. Algún icono que podría utlizarse para distinguir el módulo de los otros. Se podría utilizar en el \dashboardEF al igual que lo hace \textbf{Etiqueta descriptiva}.
		\item
			\textbf{etc}.
	\end{itemize}


	Otro detalle importante, es que la información de todos los módulos registrados serían accesibles desde cualquier módulo integrado a la arquitectura. Esto permitiría por ejemplo:

	\begin{itemize}
		\item
			Crear una vista que agrupe los módulos por tipo.
		\item
			Crear un módulo de cuentas que consulte todos los permisos que existen para todos los módulos integrados y, a través de una interfaz, asociarlos a los distintos usuarios registrados en el sistema. Con estó tendría un módulo genérico de asociación de permisos para usuarios.
		\item
			etc.
	\end{itemize}

	Notar que esta nueva funcionalidad no debería excluir ningún tipo de aplicación que se pueda desarrollar con la anterior arquitectura. Es más, la otra arquitectura también debe utilizarse para todos los módulos que se integren a esta arquitectura.



	% Caracteristicas para eCommerce
	%!TEX root = ../../../memoria.tex


% Arquitectura Framework
%!TEX root = ../../../../memoria.tex

\section{Actualizar el \frameworkPC a la nueva arquitectura}

Las ventajas:

\begin{itemize}
		\item
			\textbf{Generalizar}. Permitiría la integración de una infinidad de módulos con nuevas características sin la necesidad de hacer reestructuraciones.
		\item
			\textbf{Desacoplar}.Disminuir la superficie de contacto entre el módulo y la arquitectura, el cual será solo a traves de una interface (archivo de registro).
		\item
			\textbf{Agrupar}. Se podrían agrupar tipos de módulos. Ejemplo: módulos de pago.
		\item
			\textbf{\dashboardEF}. Permite poblar el \dashboardEF. Cada uno de los elementos en el dashboard podría tener características unicos provistas por los módulos que representan, como su nombre, utilizar el icono que lo representa. Finalmente dicho elemento sería un acceso directo al menú de configuración de dicho módulo. Lo interesante de todo esto es que sería sin agregar ningúna linea de código.
		\item
			\textbf{Módulos compatibles}. Los módulos ya creados solo deberian agregar su archivo de descripción. Todo lo demás debería permanecer básicamente igual.
		\item
			\textbf{etc}.
	\end{itemize}


% Importancia de google analytics para evaluar la usabilidad de 
%!TEX root = ../../../memoria.tex

\section{\AnalyticsCPT}
%TODO Pensar que mas agregar, por eejemplo se puede agregar lo que esta comentado aca junto con otos ejemplos
%El éxito de un sitio \ecommerceCOM está, en parte, relacionado con la facilidad en el uso.  \cite{hasan2009using}.
% Abstract. The success of an e-commerce site is, in part, related to how easy it is
% to use. This research investigated whether advanced web metrics, calculated using
% Google Analytics software, could be used to evaluate the overall usability
% of e-commerce sites, and also to identify potential usability problem areas. Web
% metric data are easy to collect but analysis and interpretation are timeconsuming.
% E-commerce site managers therefore need to be sure that employing
% web analytics can effectively improve the usability of their websites. The research
% suggested specific web metrics that are useful for quickly indicating
% general usability problem areas and specific pages in an e-commerce site that
% have usability problems. However, what they cannot do is provide in-depth detail
% about specific problems that might be present on a page. 


Hay una considerable evidencia que decisiones basadas en \analytics tienen más probabilidad de resultar correctas que aquellas basadas en la intuición. Es, al menos, mejor saber dentro de los límites de datos y análisis que \textit{creer}, \textit{pensar} o \textit{sentir}, y la mayoría de las compañias pueden beneficiarse desde desciones tomadas analíticamente. Claramente, existen circuntancias en las cuales las desiciones no pueden o no deberían realiarse analíticamente \cite{davenport2007competing}.
%There is considerable evidence that decisions based on analytics are more likely to be correct than those based on intuition.9 It’s better to know—at least within the limits of data and analysis—than to believe or think or feel, and most companies can benefit from more analytical decision making. Of course, there are some circumstances in which decisions can’t or shouldn’t be based on analytics. Some of these circumstances are described in Malcolm Gladwell’s popular book Blink, which is a paean to intuitive decision making.





% Why You Need Google Analytics

% If you owned a physical storefront, you have the ability to see you customer. You can view their habits firsthand and speak with them. Without ecommerce analytics, an online store leaves you blind to much information about your visitors and customers you would ordinarily get to see.

% Using Google Analytics can better help you understand the effectiveness of your marketing efforts, better understand your visitors and optimize your store for conversions and sales.

% Estrategia Shipping
%!TEX root = ../../../memoria.tex

\section{ Opciones de \shipping}

Cuando se inicia un sitio \ecommerceCOM es importante considerar una estrategia para \shipping dado que es un hecho que el mundo del \shipping es complejo y confuso. Sin embargo, una buena estrategia de \shipping es de vital impotancia y un elemento fundamental para cada negocio \ecommerceCOM con éxito \cite{online_shopify_shipping_toolbox}.

%Many new ecommerce entrepreneurs tend to neglect their fulfillment and shipping strategy when they launch. The reason tends to be the fact that the world of shipping is complex and confusing. However, a well educated shipping strategy is vitally important and is a core element of every successful ecommerce business.
%https://www.shopify.com/blog/14881397-ecommerce-shipping-toolbox-apps-tools-and-resources-to-help-you-streamline-your-business

	




	% El desarrollo de una arquitectura flexible que permitiera el desarrollo de grandes aplicaciones fue uno de los desafios tratados en este proyecto, sin embargo se propone el desarrollo de una componente base y flexible que funcione de \frameworkPC para el desarrollo de la mayoría de los \websitesINT que esxisten en la actualidad.

	% Lo que se propone es el desarrollo de una componente que defina una interfaz para gestionar las diversas componentes que serán acopladas para formar la aplicación particular que se desea. Esta componente \coreAS debe encargarse de detalles globales como la visibilidad de las componentes de acuerdo a permisos necesarios otorgados 

	% Como se indicó en las conclusiones, cada una de las características implementadas fueron definidas y diseádas a partir de las buenas practicas que en la actualidad existen, sin embargo dichas característcas estan lejos de ser suficientes para considerarse este proyecto como un \frameworkPC aceptable para el uso comercial. 

	% Es relevante agregar una capa de definición genérica de \packagesAS que permita agregarlos al dashboard sin la necesidad de incluir código en el \coreAS de la aplicación. De esta manera se permitirá la integración de nuevas funcionalidades de forma directa. Una vez lograda esta funcionalidad se debe desacoplar aun más los códigos existentes a través de la la creación de más \packagesAS. E incluso, 

% 	Cuando la interfaz de creación 


% -Mejorar respuestas incorrectas desde el servidor.

% - Por simplicidad momentanea, se utiliza como shipping solo el primer elemento encontrado.
%  - agregar seguridad

%  - cambiar la arquitectura. Dejar como proyecto base un manejador de packages el cual reciva

% - permitir varias tiendas 

% - Agregar cambio de themes a un click


% - How Top Retailers Show Product Images
% http://www.getelastic.com/how-top-retailers-show-product-images/


% - review : 
% http://wemakewebsites.com/blog/ecommerce-best-practices-5-must-know-tips-to-increase-your-online-sales


% Recommend related products
% A great way to increase sales is by suggesting relevant or related products during the checkout process. For example, if you have a product that works better with a complementary item, be sure to include a link to it. A customer may not realize he needs the item until it's offered to him, and it certainly doesn't hurt to make a few friendly suggestions.
% http://www.imediaconnection.com/content/36794.asp



% Don't require registration to complete checkout
% Nothing irks a customer more than being forced to create an account in order to buy something online. After all, online shopping is supposed to be easy. Sure, having an account will make future purchases easier, but it does nothing for the situation at hand. Instead, give shoppers the option of creating an account once the transaction is complete.
% http://www.imediaconnection.com/content/36794.asp

% Include product ratings and testimonials
% One of the best ways to show customers that your products are worth their while is by including ratings and testimonials in product descriptions. Someone who is initially on the fence about adding an item to their cart may be swayed after seeing positive reviews and ratings.
% http://www.imediaconnection.com/content/36794.asp








%Ver que puedo rescatar de aca para abajo


% \chapter{Caracteristicas faltantes}

% 	\section{Estabilidad del sistema}
% 		Básicamente solucionar bugs que hacen que el sistema no se comporte adecuadamente.

% 	\section{Templates}
% 		Existen ciertos elementos de las interfaces que aun no estan integradas 100\%  con los Templates. Esto implica que al cambiar de template, el color del elemento no cambia.
% 		Es un detalle menor en términos de implementación, pero relevante a la hora de diseñar un sitio \ecommerceCOM que se adeque a las necesidades.

% 	\section{Variantes de productos}
% 		El sistema debe permitir agregar variantes de productos. Un buen ejemplo de esto es cuando se venden zapatillas que son iguales salvo por la talla. No tendría sentido agregar un nuevo producto completo, solo para poder vender el mismo producto con una talla diferente.

% 	\section{Ordenes}
% 		Falta el manejo de las ordenes de compra. Las interfaces estan listas, solo falta trabajar en el backend.

% 	\section{Proceso Shipping completo}
% 		Actualmente es posible ver la vista de shipping y sus diferentes elementos sin nigún tipo de inconveniente. Sin embargo el sistema aun carece de \textit{guiar al usuario}. Por ejemplo el estado global del sistema no se actualiza automáticamente aun.
% 		Básicamente lo que falta acá es terminar con la máquina de estados que actualmente define este proceso.

% 	\section{Sistema de Pago}
% 		El sistema no tiene configurado un método para agregar sistema de pago. La idea tras esto es crear un package que permita administrar sistemas de pago. Cada nuevo sistema de pago será visto como un nuevo package.
% 		Como mínimo, se implementará el administrador de sistemas de pago junto con un sistema de pago como Pay Pal, Google Wallet, etc.

% 	\section{Agregar archivos de configuración}
% 		Agregar ciertos archivos que permitan hacer configuraciones más sencillas del sistema sin tener la necesidad de involucrarse directamente con el código.
% 		Por ejemplo la máquina de estados del proceso shipping debe estar descrito utilizando un archivo.

% 	\section{Analytics}

% 		Agregar alguna herramienta para realizar analytics, tal como Google Analytics.
% 		Estas herramientas son de vital importancia para hacer marketing inteligente y desiciones de negocio.

% 	\section{Logging out de otras secciones}
% 		%Today, it's important to provide users with the ability to log out of not only their current session, but also any sessions they may have open from other devices. Otherwise what is a user to do if their mobile phone is lost or stolen, with a live login session still on the device? With Meteor Accounts this is as simple as providing a button that when clicked calls Meteor.logoutOtherClients.
% 		En la actualidad es importante permitir a los usuarios no solo cerrar la sesión actual, si no tambien la de todos los dispositivos que tengan una sesión abierta. De esta manera solucionar el típico problema de un teléfono robado o perdido con una sesión activa.






