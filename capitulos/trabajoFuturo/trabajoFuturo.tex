%!TEX root = ../../memoria.tex
\chapter{Trabajos Futuros}\label{cap:trabajos_futuros}

	El desarrollo de una arquitectura flexible que permitiera el desarrollo de grandes aplicaciones fue uno de los desafios tratados en este proyecto, sin embargo se propone el desarrollo de una componente base y flexible que funcione de \frameworkPC para el desarrollo de la mayoría de los \websitesINT que esxisten en la actualidad.

	Lo que se propone es el desarrollo de una componente que defina una interfaz para gestionar las diversas componentes que serán acopladas para formar la aplicación particular que se desea. Esta componente \coreAS debe encargarse de detalles globales como la visibilidad de las componentes de acuerdo a permisos necesarios otorgados 

	Como se indicó en las conclusiones, cada una de las características implementadas fueron definidas y diseádas a partir de las buenas practicas que en la actualidad existen, sin embargo dichas característcas estan lejos de ser suficientes para considerarse este proyecto como un \frameworkPC aceptable para el uso comercial. 

	Es relevante agregar una capa de definición genérica de \packagesAS que permita agregarlos al dashboard sin la necesidad de incluir código en el \coreAS de la aplicación. De esta manera se permitirá la integración de nuevas funcionalidades de forma directa. Una vez lograda esta funcionalidad se debe desacoplar aun más los códigos existentes a través de la la creación de más \packagesAS. E incluso, 

	Cuando la interfaz de creación 


-Mejorar respuestas incorrectas desde el servidor.

- Por simplicidad momentanea, se utiliza como shipping solo el primer elemento encontrado.
 - agregar seguridad

 - cambiar la arquitectura. Dejar como proyecto base un manejador de packages el cual reciva

- permitir varias tiendas 

- Agregar cambio de themes a un click


- How Top Retailers Show Product Images
http://www.getelastic.com/how-top-retailers-show-product-images/


- review : 
http://wemakewebsites.com/blog/ecommerce-best-practices-5-must-know-tips-to-increase-your-online-sales


Recommend related products
A great way to increase sales is by suggesting relevant or related products during the checkout process. For example, if you have a product that works better with a complementary item, be sure to include a link to it. A customer may not realize he needs the item until it's offered to him, and it certainly doesn't hurt to make a few friendly suggestions.
http://www.imediaconnection.com/content/36794.asp



Don't require registration to complete checkout
Nothing irks a customer more than being forced to create an account in order to buy something online. After all, online shopping is supposed to be easy. Sure, having an account will make future purchases easier, but it does nothing for the situation at hand. Instead, give shoppers the option of creating an account once the transaction is complete.
http://www.imediaconnection.com/content/36794.asp

Include product ratings and testimonials
One of the best ways to show customers that your products are worth their while is by including ratings and testimonials in product descriptions. Someone who is initially on the fence about adding an item to their cart may be swayed after seeing positive reviews and ratings.
http://www.imediaconnection.com/content/36794.asp


% Mejoras en la arquitectura
%!TEX root = ../../memoria.tex

\section{Mejora en la arquitectura}
%TODO Falta
Durante el desarrollo de la aplicación pude notar el fuerte concepto que representa \packageAS para \meteorNAME. Esto me sugirio considerar mejoras en la arquitectura las cuales no fueron posibles dado el tiempo que esta considerado para el desarollo de una memoria.
Básicamente el cambio consiste en desarrollar un proyecto base que administre todos los \packageAS



% Importancia de google analytics para evaluar la usabilidad de 
%!TEX root = ../../../memoria.tex

\section{\AnalyticsCPT}
%TODO Pensar que mas agregar, por eejemplo se puede agregar lo que esta comentado aca junto con otos ejemplos
%El éxito de un sitio \ecommerceCOM está, en parte, relacionado con la facilidad en el uso.  \cite{hasan2009using}.
% Abstract. The success of an e-commerce site is, in part, related to how easy it is
% to use. This research investigated whether advanced web metrics, calculated using
% Google Analytics software, could be used to evaluate the overall usability
% of e-commerce sites, and also to identify potential usability problem areas. Web
% metric data are easy to collect but analysis and interpretation are timeconsuming.
% E-commerce site managers therefore need to be sure that employing
% web analytics can effectively improve the usability of their websites. The research
% suggested specific web metrics that are useful for quickly indicating
% general usability problem areas and specific pages in an e-commerce site that
% have usability problems. However, what they cannot do is provide in-depth detail
% about specific problems that might be present on a page. 


Hay una considerable evidencia que decisiones basadas en \analytics tienen más probabilidad de resultar correctas que aquellas basadas en la intuición. Es, al menos, mejor saber dentro de los limites de datos y análisis que \textit{creer}, \textit{pensar} o \textit{sentir}, y la mayoría de las compañias pueden beneficiarse desde desciones tomadas analíticamente. Claramente, existen circuntancias en las cuales las desiciones no pueden o no deberían realiarse analíticamente \cite{davenport2007competing}.
%There is considerable evidence that decisions based on analytics are more likely to be correct than those based on intuition.9 It’s better to know—at least within the limits of data and analysis—than to believe or think or feel, and most companies can benefit from more analytical decision making. Of course, there are some circumstances in which decisions can’t or shouldn’t be based on analytics. Some of these circumstances are described in Malcolm Gladwell’s popular book Blink, which is a paean to intuitive decision making.





% Why You Need Google Analytics

% If you owned a physical storefront, you have the ability to see you customer. You can view their habits firsthand and speak with them. Without ecommerce analytics, an online store leaves you blind to much information about your visitors and customers you would ordinarily get to see.

% Using Google Analytics can better help you understand the effectiveness of your marketing efforts, better understand your visitors and optimize your store for conversions and sales.

% Estrategia Shipping
%!TEX root = ../../../memoria.tex
\section{\shippingEF}\label{chapter:solucionimplementada:section:shipping}

	El proceso consiste en 6 pasos que deben cumplirse en orden estricto.

	En la parte superior de esta vista( \refFigura{figure:shipping:global_status}), se puede ver en todo momento el estado actual del \workflowCPT del proceso \shippingEF.

	\begin{figure}[H]
		\centering
		\includegraphics[width=0.7\textwidth]{figuras/shipping/global_status.png}
		\caption{Estado actual del \workflowCPT del \shippingEF. En este caso se encuentra en la selección de la dirección.}
		\label{figure:shipping:global_status}
	\end{figure}

	Adicionalmente a eso, la vista de \shippingEF muestra en detalle todos los pasos (\refFigura{figure:shipping:steps}). Cada uno de estos pasos se define como:

	\begin{description}
		\item[Account] \hfill \\
			Corresponde al proceso de autenticarse en la plataforma.
		\item[Address Details] \hfill \\
			Paso para la selección de una de las direcciones configuradas. Si no se tiene ningúna dirección agregada, el sistema muestra inmediatamente el formulario para la creación de una nueva dirección (\refFigura{figure:shipping:form_address}).
		\item[Shipping Options] \hfill \\
			Permite seleccionar uno de los métodos de envío disponibles. Estos son configurados por el administrador 
		\item[Review] \hfill \\
			Muestra un resumen de la información agregada al carro de compra. 
		\item[Complete] \hfill \\
			Paso de selección de método de pago.
	\end{description}


	\begin{figure}[H]
		\centering
		\includegraphics[width=0.8\textwidth]{figuras/shipping/steps.png}
		\caption{Detalle de todos los pasos del \workflowCPT \shippingEF.}
		\label{figure:shipping:steps}
	\end{figure}

	%TODO: modificar este texto despues
	En la \refFigura{figure:shipping:steps} se observa que los pasos 3 y 5 tienen un mensaje advirtiendo que no hay métodos de \shippingEF y métodos de pago configurados. Esta configuración debe ser realizada por el o los administradores de la tienda. Pero actualmente no están disponibles las interfaces para realizar estas acciones.

	\begin{figure}[H]
		\centering
		\includegraphics[width=0.8\textwidth]{figuras/shipping/step_address.png}
		\caption{Seleccionar dirección en proceso de \shippingEF.}
		\label{figure:shipping:step_address}
	\end{figure}

	\begin{figure}[H]
		\centering
		\includegraphics[width=0.6\textwidth]{figuras/shipping/form_address.png}
		\caption{Despliegue automático del formulario de direcciones cuando no se tienen ninguna dirección agregada.}
		\label{figure:shipping:form_address}
	\end{figure}


	En el caso de tener direcciones configuradas, pero se desee hacer el envío a una nueva dirección, es posible crearla apretando el botón \textit{Add Address} y un formulario se desplegará (\refFigura{figure:shipping:form_add_address}). Este formulario es equivalente al de la \refFigura{figure:shipping:form_address}, pero este incluye la opción de cancelar la creación de la dirección.

	\begin{figure}[H]
		\centering
		\includegraphics[width=0.6\textwidth]{figuras/shipping/form_add_address.png}
		\caption{Formulario de creación de una nueva dirección. Se diferencia del formulario de la \refFigura{figure:shipping:form_address} en que este incluye el botón Cancel.}
		\label{figure:shipping:form_add_address}
	\end{figure}
