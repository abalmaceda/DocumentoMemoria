
\chapter{Diseño Solución}\label{cap:diseño_solucion}

Una aplicación \webINT es aquella que corre sobre un \webINT \browserINT, el cual obtiene los datos que procesa a través de un protocolo estándar de comunicación llamado \httpNAME que opera en la \applayer, la cual es parte de \osimodel.

\osimodel es un modelo estándar de arquitectura que constituye el \frameworkPC para el desarrollo de protocolos estándar para la interconexión de sistemas. Dicho \frameworkPC esta compuesto por 7 \layers que separan las diferentes funcionalidades necesarias(\refFigura{figure:framework_layers_definition})\cite{inbook_osi_reference_model}, de esta forma se divide el problema en partes mas pequeñas.

\begin{figure}[h!]
	\begin{tikzpicture}[node distance=3pt,
		blueb/.style={
	  		draw=white,
	  		fill=mybluei,
	  		rounded corners,
	  		text width=2.5cm,
	  		font={\sffamily\bfseries\color{white}},
	  		align=center,
	  		text height=12pt,
	  		text depth=9pt},
		greenb/.style={blueb,fill=mygreen},
	]

	\node[blueb,text width=13cm+26pt] (APP) {\applayer};
	\node[blueb,below=of APP,text width=13cm+26pt] (PRES) {\preslayer};
	\node[blueb,below=of PRES,text width=13cm+26pt] (SESSION) {\sessionlayer};
	\node[blueb,below=of SESSION,text width=13cm+26pt] (TRANS) {\translayer};
	\node[blueb,below=of TRANS,text width=13cm+26pt] (NETWORK) {\networklayer};
	\node[blueb,below=of NETWORK,text width=13cm+26pt] (LINK) {\datalinklayer};
	\node[blueb,below=of LINK,,text width=13cm+26pt] (PHYSI) {\physilayer};
	
	\begin{pgfonlayer}{background}
	\draw[blueb,draw=black,fill=mybluei!30] 
	  ([xshift=-8pt,yshift=8pt]current bounding box.north west) rectangle 
	  ([xshift=8pt,yshift=-8pt]current bounding box.south east);
	\end{pgfonlayer}
\end{tikzpicture}

\caption{Arquitectura \osimodel }
	\label{figure:framework_layers_definition}
\end{figure}

La idea básica de separar en \layers, es que cada una de estas agrega valor a los servicios provistos por el conjunto inferior de \layers de tal manera que las \layer que esta en la cúspide del \stackAS ofrece el conjunto de servicios necesarios para ejecutar aplicaciones distribuidas. Cada una de estas 7 \layers representa a su vez un \frameworkPC que define sus funcionalidades propias.

\begin{figure}[h!]
	\begin{tikzpicture}[node distance=10pt,
		blueb/.style={
	  		draw=white,
	  		fill=mybluei,
	  		rounded corners,
	  		text width=2.5cm,
	  		font={\sffamily\bfseries\color{white}},
	  		align=center,
	  		text height=8pt,
	  		text depth=5pt},
		greenb/.style={blueb,fill=mygreen},
	]

	\node[blueb,text width=5cm+10pt] (APP) {\applayer};
	\node[below=of APP] (dummy1) {};
	\node[right=of dummy1] (dummy2) {};
	
	\node[above=of PRES] (dummy3) {};
	\node[right=of dummy3] (dummy4) {};
	\node[blueb,below=of APP,text width=5cm+10pt] (PRES) {\preslayer};
	\path[->] (dummy1) edge node {} (dummy3);
	\path[->] (dummy2) edge node {} (dummy4);
	
	\node[blueb,below=of PRES,text width=5cm+10pt] (SESSION) {\sessionlayer};
	\path[->] (PRES.20) edge node {} (SESSION);
	\path[->] (SESSION.-20) edge node {} (PRES);
	
	\node[blueb,below=of SESSION,text width=5cm+10pt] (TRANS) {\translayer};
	\path[->] (SESSION.20) edge node {} (TRANS);
	\path[->] (TRANS.-20) edge node {} (SESSION);
	
	\node[blueb,below=of TRANS,text width=5cm+10pt] (NETWORK) {\networklayer};
	\path[->] (NETWORK.20) edge node {} (TRANS);
	\path[->] (TRANS.-20) edge node {} (NETWORK);
	
	\node[blueb,below=of NETWORK,text width=5cm+10pt] (LINK) {\datalinklayer};
	\path[->] (LINK.20) edge node {} (NETWORK);
	\path[->] (NETWORK.-20) edge node {} (LINK);
	
	\node[blueb,below=of LINK,,text width=5cm+10pt] (PHYSI) {\physilayer};
	\path[->] (PHYSI.20) edge node {} (LINK);
	\path[->] (LINK.-20) edge node {} (PHYSI);
	
%	\begin{pgfonlayer}{background}
%	\draw[blueb,draw=black,fill=mybluei!30] 
%	  ([xshift=-8pt,yshift=8pt]current bounding box.north west) rectangle 
%	  ([xshift=8pt,yshift=-8pt]current bounding box.south east);
%	\end{pgfonlayer}
\end{tikzpicture}

\caption{Arquitectura \osimodel }
	\label{figure:framework_layers_interaction}
\end{figure}

Lo que se plantea es desarrollar un \frameworkPC que se encontrará inmediatamente por sobre el \osimodel (\refFigura{figure:framework_above_osimodel}) para facilitar el desarrollo de aplicaciones, productos y soluciones específicas \ecommerce utilizando un ambiente de \softwarePC reusable y universal a través de funcionalidades particulares que son parte de una gran plataforma.

\begin{figure}[h!]
	\begin{tikzpicture}[node distance=3pt,
		blueb/.style={
	  		draw=white,
	  		fill=mybluei,
	  		rounded corners,
	  		text width=2.5cm,
	  		font={\sffamily\bfseries\color{white}},
	  		align=center,
	  		text height=12pt,
	  		text depth=9pt},
		greenb/.style={blueb,fill=mygreen},
	]

		\node[blueb, draw=black, fill=myblueii] (SLCN1) {Solución 1};
		\node[blueb, draw=black, fill=myblueii,right=of SLCN1] (SLCN2) {Solución 2};
		\node[right=of SLCN2] (dummy) {};
		\node[blueb, draw=black, fill=myblueii,right=of SLCN2] (SLCN3){Solución 3};
		\node[blueb, draw=black, fill=myblueii,right=of SLCN3] (SLCN4) {Solución 4};
		\begin{pgfonlayer}{background}
			\draw[blueb,draw=black,fill=mybluei!30] 
  				([xshift=-8pt,yshift=8pt]current bounding box.north west) rectangle 
  				([xshift=8pt,yshift=-8pt]current bounding box.south east);
		\end{pgfonlayer}
\node[blueb, draw=black, fill=myblueii, below=22pt of dummy,text width=13cm+26pt] (ECOMM) {\frameworkname};

		\node[blueb, draw=black, fill=myblueii, below=of ECOMM, text width=13cm+26pt] (OSI) {\osimodel};

\end{tikzpicture}

\caption{Arquitectura general de la plataforma \ecommerce }
	\label{figure:framework_above_osimodel}
\end{figure}

%that actual software is an application running on your PC. It doesn't really “reside” at the application layer. Rather, it makes use of the services offered by a protocol that operates at the application layer, which is called the Hypertext Transfer Protocol (HTTP). The distinction between the browser and HTTP is subtle, but important.

%In computer programming, a software framework is an abstraction in which software providing generic functionality can be selectively changed by additional user-written code, thus providing application-specific software. A software framework is a universal, reusable software environment that provides particular functionality as part of a larger software platform to facilitate development of software applications, products and solutions. Software frameworks may include support programs, compilers, code libraries, tool sets, and application programming interfaces (APIs) that bring together all the different components to enable development of a project or solution.
