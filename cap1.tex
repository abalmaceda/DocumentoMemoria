%%%%%%%%%%%%%%
%					%
%	INTRODUCCIÓN	%
%					%
%%%%%%%%%%%%%%
\chapter{Introducción}\label{cap:intro}

%Historia de \ecommerce

Una de las actividad mas populares en la \web es comprar. La razón es por que tiene un encanto especial - puedes comprar en tiempo libre, cualquier momento, incluso usando pijamas. Literalmente cualquiera puede tener páginas para mostrar sus productos y servicios.

La historia de \ecommerce se remonta a la invención de la básica noción de \“vender y comprar\”, electricidad, cables, computadores, \textit{modems}, y la \internet. \ecommerce se hace posible en 1991 cuando la \internet estuvo disponible para uso comercial. Desde entonces miles de negocios se han establecido en sitios \web.

En un inicio, el termino \ecommerce se refería al proceso de ejecución  de transacciones electrónicas comerciales con la ayuda de tecnologías lideres tales como Electronic Data Interchange (EDI) y Electronic Funds Transfer (EFT) las cuales dieron la oportunidad a los usuarios para intercambiar información de negocios y realizar transacciones electrónicas. La oportunidad para utilizar estas tecnologías surgió a finales de 1970s y permitió a los negocios de las compañías y organizaciones enviar documentación electrónica comercial.

Aunque la \internet comenzó a ganar popularidad  entre el publico general en 1994, tomó aproximadamente cuatro años desarrollar protocolos de seguridad ( por ejemplo \http) y DSL los cuales permitieron un acceso rápido y conexiones persistentes a la \internet. En 2000 un gran número de empresas comerciales en los Estados Unidos y Europa Occidental presentaron sus servicios en la \www. En ese entonces el significado de la palabra \ecommerce fue cambiado. Las personas comenzaron a utilizar el termino \ecommerce como el proceso de compra de productos y servicios disponibles en \internet utilizando conexiones seguras y servicios de pago electrónico. Aunque el colapso de \dotcom en 2000 dirigió a desafortunados resultados y muchas compañías \ecommerce desaparecieron, el \textbf{\textit{brick and mortar} retailers} reconocieron las ventajas del comercio electrónico y comenzaron a la integración de tales características a sus sitios \web. Para finales de 2001, el modelo de negocio mas grande de \ecommerce, Businnes-to-Bussines (B2B), había ganado al rededor de \$700 billones en transacciones.

-------------------- AQUI UNA FOTO -------------------

Acorde a toda información disponible, las ventas \ecommerce continuaron creciendo en los siguientes años y, a finales de 2007 ventas \ecommerce representaron el 3.4\% de las ventas totales en el mundo.

\ecommerce tiene muchas ventajas sobre \textit{brick and mortar} tiendas y catálogos de venta por correo. Los consumidores pueden fácilmente buscar a través de una base de datos muy extensa de productos y servicios. Pueden ver los precios reales, definir una orden de compra en varios días y enviar un correo como un \wishlist esperando que alguien pague por sus productos seleccionados. Los clientes pueden comparar precios con un simple \textit{click} del \textit{mouse} y comprar los productos seleccionados al mejor precio.

Proveedores \online, en sus turnos, también tienen claras ventajas. La \web y sus motores de búsqueda proveen una manera para encontrar clientes sin campañas de publicidad costosas. Incluso tiendas \online pequeñas pueden alcanzar mercados globales. 
La tecnología \web también permite realizar un seguimiento sobre las preferencias de los clientes para ofrecer una comercialización personalizada.

%La historia de \ecommerce is impensable sin Amazon e Ebay los cuales estuvieron entre las primeras compañías que permitían transacciones electrónicas. Gracias a sus fundadores ahora tenemos un sector considerable de \ecommerce y disfrutar de comprar y vender gracias a la Internet. Actualmente hay 5 de los mas grandes y mas famosos \textit{worldwide internet retailers}: Amazon, Dell, Staples, Office Depot y Hewlett Packard. De acuerdo a las estadísticas, las categorías de productos mas vendidos en \textit{Worl Wide Web} son música, libros, computadores, artículos de oficina y otros dispositivos electrónicos.
%
%Amazon.com, Inc. es uno de las mas famosas compañías \ecommerce  para vender productos sobre la Internet. Después del colapso \textit{dot-com} Amazon perdió su posición de modelo de negocio exitoso, sin embargo, en 2003 la compañía hizo su primer año con utilidades el cual fue el primer paso para el desarrollo futuro.
%
%Al principio Amazon.com fue considerado como una tienda \textit{online} de libros, pero con el tiempo una variedad de productos electrónicos fueron agregados, software, DVDs, juego de video, CDs de música, MP3s, prendas de vestir, calzado, productos de salud, etc. El nombre original de la compañía fue Cadabra.com, pero rápidamente después que se volviera popular Internet Bezos decidió cambiar el nombre de su negocio a “Amazon” después del río voluminoso mas grande del mundo. En 1999 Jeff Bezos fue nombrado como la persona del año por Time Magazine en reconocimiento al éxito de la compañía. Aunque la sede principal de la empresa se encuentra en USA, WA, Amazon ha establecido sitios \textit{web} separados en otros países tales como United Kingdom, Canada, France, Germany, Japan, y China. La compañía apoya y opera \textit{retail web sites} para muchos negocios famosos, incluyendo Marks \& Spencer, Lacoste, la NBA, Bebe Stores, Target, etc.
%
%Amazon es uno de los primeros negocios \ecommerce en establecer un programa de marketing para los afiliados, y actualmente la compañía obtiene cerca del 40\% de sus ventas desde afiliados y vendedores de terceras partes que lista y vende productos en el sitio \textit{web}. En 2008 Amazon penetro en el cine y actualmente esta patrocinando la película \textit{“The Stolen Child”} con \textit{20th Century Fox}.
%
%Acorde a las investigaciones en 2008, el dominio Amazon.com a traído cerca de 615 millones de clientes cada año. La característica mas popular del sitio \textit{web} es el \textit{review sistem}, i.e., la habilidad de los visitantes de presentar sus \textit{reviews} y \textit{rate} cualquier producto en un \textit{raiting scale} de uno a 5 estrellas. Amazon.com es también \textit{well-know} por su \textit{clear and user-friendly} avanzado sistema de búsqueda que permite a los visitantes buscar por \textit{keywords} en el texto completo de muchos libros en la base de datos.
%
%Otra compañía ha contribuido mucho en el proceso de desarrollo de \ecommerce es Dell Inc., una compañía americana posicionada en Texas, que se sitúa en el tercer lugar de ventas de computadoras después de Hewllett-Packard y Acer.
%
%Lanzado en 1994 como una pagina estática, Dell.com ha realizado rápidos pasos, y para finales de 1997 fue la primera compañía en lograr el \textit{record} de un millón de ventas \textit{online}. Su única estrategia de ventas sobre la \textit{World Wide Web} sin \textit{retail outlets} y sin intermediarios ha sido admirado por un sin fin de clientes e imitado por un gran numero de \ecommerce \textit{businesses}. El factor de éxito de Dell es que Dell.com permite a los clientes elegir y controlar, i.e., visitantes pueden explorar el sitio y ensamblar PC pieza por pieza eligiendo el mas mínimo componente basado en sus presupuestos y requerimientos. De acuerdo a las estadísticas, aproximadamente la mitad de las ganancias de las compañías proviene desde su sitio \textit{web}.
%
%En 2007, Fortune magazine \textit{ranked} Dell como la \textit{34th-largest} compañía en \textit{Fortune 500 list}, y \textit{8th} en su \textit{Top 20 list} anual de las mas exitosas y admiradas compañías en USA en reconocimiento al \textit{bussiness model} de la compañía.

La historia de \ecommerce es nueva, un mundo virtual que esta evolucionando de acuerdo a las ventajas del cliente. Es un mundo que todos construyen en conjunto ladrillo por ladrillo, estableciendo una base segura para las nuevas generaciones.

\subsection{\ecommerce en la actualidad}

En la actualidad, \ecommerce es una experiencia destacable. Transformo las compras tradicionales mas allá de lo reconocible. La experiencia es mucho mejor que cualquier otra manera de comprar, seduciendo a una gran cantidad de \ecommerce \textit{lovers}.

Si hace algunos años atrás \ecommerce fue una palabra de moda, ahora se ha convertido en la orden del día. Al parecer las personas compran literalmente en cualquier parte - en sus lugares de trabajo durante el almuerzo, en las horas punta cuando no hay nada mas por hacer salvo encender el computador y comenzar a navegar.

En el presente, \ecommerce ha ganado tanta popularidad debido a que su tecnología subyacente esta evolucionando a pasos agigantados. Incluso se esta ofreciendo \textit{"feel"} el producto con un \textit{mouse 3D} para comprender mejor su forma, tamaño y textura. ¿Para que salir cuando todo lo que se debe hacer es realizar un pedido, elegir la forma de envío, pararse y esperar hasta que la orden sea entregada hasta la puerta?

\ecommerce hoy ofrece tanto lujo que incluso las tiendas convencionales han encendido las alarmas. Aunque, cada uno esta de acuerdo que hay un gran camino para que \ecommerce reemplace las tiendas, la posibilidad existe que ocurra en el futuro. \ecommerce del cual somos actualmente testigos trae tanta aventura a nuestras vidas que es disfrutado por toda la comunidad \online.

%En la actualidad, \textit{e-Commerce} tiene impuestos ...................

\subsection{\ecommerce en el futuro}


Expertos predicen un prometedor y glorioso futuro para \ecommerce. En un presumible futuro, \ecommerce 


\section{Motivación}\label{cap:intro:motivacion}

Acá va la motivación...

\begin{itemize}
	\item \textbf{Shopping-Cart}:
	
	\item \textbf{Reclamos ciudadanos}: Un sistema para realizar y gestionar reclamos que estén relacionado principalmente temas sociales con el fin de alertar a la brevedad a los organismos públicos responsables. Ejemplos de uso serian: Calles en mal estado, semáforos en mal estado, semáforos que no tengan un tiempo suficiente para cruzar, cruces peligrosos para peatones, aceras en mal estado, lugares de ruido excesivo, lugares de robos comunes, plazas en mal estado. La idea básica es crear una queja, y que los ciudadanos que consideren que dicha queja los representa, unirse a ella para lograr un volumen crítico para que las autoridades sean consientes del impacto que genera dicha queja.
	
	\item \textbf{Reservas horas medicas:} Un sistema para gestionar las horas medicas que muestre sugerencias de acuerdo a la distancia que se encuentran los especialistas desde la \textit{geo-posicion} del sistema que realiza la solicitud. 
	
	\item \textbf{Consultas de libros en una biblioteca:} Un sistema para consultar la disponibilidad, la cantidad, la posibilidad de reservarlo, e incluso cuando sera devuelto.
	\item \textbf{Catálogo de Moteles}: Un catálogo de moteles de acuerdo a la \textit{geo-posicion} pueden ser listados para determinar las mejores opciones disponibles. Adicionalmente se podría mostrar los horarios disponibles de las habitaciones, e incluso poder generar reservas y pagar la habitación. De esta manera se optimiza el tiempo de los clientes, y los recursos de los proveedores.
	
	\item \textbf{Catalogo de Medicamentos}: Un catalogo completo de los medicamentos oficialmente aceptados podrían ser listados para entregar información relevante de acuerdo a las dosis así como de efectos adversos, si es necesaria una receta, etc. Otra cosa interesante, es dar la opción de mostrar todos los remedios genéricos que existen en el mercado. El escenario ideal seria ademas mostrar el precio de estos, así como la distancia a las droguerías en donde el producto se encuentra disponible.
	
	\item \textbf{Reserva de horas en Registro Civil}: Al menos en Chile, realizar un tramite en el registro civil es sinónimo de una perdida de tiempo absurda solo para ser atendido. Si la reserva de hora puede ser realizada desde cualquier sitio, el sistema se volvería drásticamente mas eficiente.
	
	\item \textbf{Catalogo de Fiestas}: Un catalogo con las fiestas disponibles tanto a la brevedad posible, como en el futuro. Que muestre información relevante 
	
\end{itemize}

Todas estas situaciones tienen algo en común, y es que los usuarios desean consultar información para tomar una decisión. El sistema idealmente permite contrastar datos entre diferentes proveedores de servicios similares, así como de dejar información clara y relevante, para que futuros usuarios puedan tomar una decisión aun mas informados.

Como se puede concluir, el modelo de negocio de estas situaciones es en su mayoría el mismo, solo cambiado la presentación para hacerlo \textit{ad-hoc} a la situación.

\section{Alcances y Objetivo General}\label{cap:intro:alcances}
Desarrollar un código base que permita desarrollar una familia aplicaciones. Dicha aplicación se utilizara para desarrollar 3 ejemplos a fin de demostrar su utilidad.

\section{Objetivos Específicos}\label{cap:intro:objetivos}
%Dentro de los objetivos específicos se encuentran:
\begin{itemize}
	\item Determinar la tecnología mas adecuada para el desarrollo del proyecto
	\item Determinar la arquitectura adecuada para el desarrollo de software genérico
	\item Desarrollar un \ecommerce
	\item Objetivo 4
\end{itemize}

\section{Caracteristicas deseables}\label{cap:intro:alcance}

\begin{table}[h!]
    \tiny
   
\begin{tabular}{ |L{0.6\paperwidth}|L{0.1\paperwidth}|}
\hline
	\task &
	Tiempo
	
\\ \hline
	\textbf{ Add:} Arrange order of product's variants in your shop with drag and drop.&
	
\\ \hline
	\textbf{ \textit{Drag \& Drop} manejo de variedad}: Organizar el orden de la variedad de productos en la tienda con \textit{drag and grop}.&
	%\textbf{ Drag \& Drop Variant Management:} Arrange order of product's variants in your shop with drag and drop.&
		
\\ \hline
	\textbf{ \textit{Drag \& Drop Merchandising}}: Organizar el orden de los productos en la tienda con \textit{drag and grop}.&
	%\textbf{ Drag \& Drop Merchandising:} Arrange order of products in your shop with drag and drop.&
		
\\ \hline
	\textbf{ Device Agnostic:} Optimized experience for all mobile, tablet, and desktop devices.&
		
\\ \hline
	 \textbf{Edición del campo \textit{Inline}}: Agregar y editar contenido de la tienda \textit{clicking} cualquier \textit{text field}.&
	% \textbf{ Inline Field Editing:} Add and edit your shop's content by clicking any text field.&
		
\\ \hline
	\textbf{ Google Analytics:} Out of the box Google Analytics event tracking.&
		
\\ \hline
	\textbf{ PayPal Integration:} Supports the ability to use PayPal checkout.&
		
\\ \hline
	\textbf{\textit{Real-time item-updates}}: Cambios realizados a la tienda son instantáneamente observados por los compradores (sin refrescar la página).&
	%\textbf{ Real-time Reactive:} Changes made to your shop are instantly seen by your shoppers (without page reloads).&
		
\\ \hline
	\textbf{Clonar productos existentes}: Crear nuevos productos clonando cualquer producto existente.&
	%\textbf{ Clone Existing Products:} Create new products by cloning any existing products.&
		
\\ \hline
	\textbf{ Multiple Images Per Product Variant:} Add multiple product images per product option.&
		
\\ \hline
	\textbf{ Recursive Tag Taxonomy:} Uses recursive tag taxonomy for categorization.&
		
\\ \hline
	 \textbf{ Clone Product Variants:} Create new product variants by cloning any existing product variant.&
	
\\ \hline
	\textbf{ Fully Open Source:} Node.js and Meteor. Completely Open source. All day. Every day.&
	
\\ \hline
	\textbf{ Quantity per Variant:} Inventory support by product variant.&
	
\\ \hline
	\textbf{ Published or Hidden Status:} Ability to have products in a "draft" state before publishing.&
	
\\ \hline
	\textbf{ Product Details:} Supports additional product details in key/value list.&
	
\\ \hline
	\textbf{ SEO Hashtags:} Uses social media hashtags for product urls for simple SEO+social media tracking.&
	
\\ \hline
	\textbf{ One Page Checkout:} Checkout all done on one page.&
	
\\ \hline
	\textbf{ Shop Analytics:} Integrated tracking framework for integration to any analytics system&
	
\\ \hline
	\textbf{ Docker.io Ready:} Build, Ship and Run anywhere.&
	
\\ \hline
	 \textbf{ Backorders:} Shoppers can purchase products even when quantity runs out.&
	
\\ \hline
	\textbf{ i18n and l10n:} Internationalization and localization to translate and localize all content for the world.&
	
\\ \hline
	\textbf{ Variant Management:} Easily manage product options (e.g. size/color) and related product photos.&
	
\\ \hline
	\textbf{ App Gallery:} Select from a gallery of apps to extend your shop by adding your favorite tools.&
	
\\ \hline
	\textbf{ Social Media Integration:} Integrated custom product social media messaging (FB, Twitter, Pinterest, Instagram).&
	
\\ \hline
	\textbf{ Custom Domains:} Use your own domain names.&
	
\\ \hline
	\textbf{ Flat Rate Tax Management:} Manage tax rules for your store.&
	
\\ \hline
	\textbf{ Additional Payment Methods:} Support for more payment methods (providers TBD).&
	
\\ \hline
	\textbf{ Global Shipping:} Ship your products around the globe (carriers TBD).&
	
\\ \hline
	\textbf{ User Management:} Invite users and grant permissions.&
	
\\ \hline
	\textbf{ Email Templates:} Manage email templates for your shop.&
	
\\ \hline
	\textbf{ Hero Manager:} Inline management for hero sections on your shop.&
	
\\ \hline
	\textbf{ Search:} Auto-suggest, keyword product search.&
	
\\ \hline
	\textbf{ Braintree Integration:} Braintree Integration.&
	
\\ \hline
	\textbf{ Stripe Integration:} Use Stripe for payments.&
	
\\ \hline
	\textbf{ Native Mobile Apps:} Build and deploy iOS, Android native Reaction apps.&
	
\\ \hline
	\textbf{ Multi-Currency Support:} Support for additional currencies beyond US Dollars.&
	
\\ \hline
	\textbf{ RTL Localization:} Right to left language support.&
	
\\ \hline
	\textbf{ Revision Control:} Revision control with rollback for all edits.&
	
\\ \hline
	\textbf{ History Logging:} Full insight to all actions performed.&
	
\\ \hline
	\textbf{ Cash on Delivery Payments:} Cash on delivery payment methods.&
	
\\ \hline
	\textbf{ API:} Support for both HTTP API and Meteor DDP.&
	
\\ \hline
	\textbf{ Product Inheritance:} Manage product pricing, promotions, visibility through parent-child-clone inheritance.&
	
\\ \hline
	\textbf{ Coupon Codes:} Coupon management and tracking for discounts.&
	
\\ \hline
	\textbf{ Returns and Refunds:} Track, manage, and analytics on returns and exchanges.&
	
\\ \hline
	\textbf{ Multi-Vendor:} Multiple vendors with review, publish, drop shipping.&
	
\\ \hline
	\textbf{ Flexible Tax Management:} Manage and customize tax rules.&
	
\\ \hline
	\textbf{ Subscription Products:} Subscription based product types.&
	
\\ \hline
	\textbf{ Order Entry and Editing:} Adminstrator addition and editing of orders.&
	
\\ \hline
	\textbf{ Embed Social Content:} Embed reviews, tweets, and other social content.&
	
\\ \hline
	\textbf{ Bitcoin Integration:} Ability to accept Bitcoin payments in your shop.&
	
\\ \hline
	\textbf{ Amazon Payments Integration:} Use Amazon for payments.&
	
\\ \hline
	\textbf{ Promotions:} Ability to manage and track promotions by channels, events, and more.&
	
\\ \hline
	\textbf{ Google Wallet Integration:} Use Google Wallet for payments.&
	
\\ \hline
	\textbf{ Shipwire Integration:} Ability to use Shipwire for order fulfillment.&
	
\\ \hline
	\textbf{ ShipStation Integration:} Ability to use ShipStation for order fulfillment.&
	
\\ \hline
	\textbf{ MailChimp Integration:} Use MailChimp to collect and manage emails.&
	
\\ \hline
	\textbf{ Actionable Analytics:} Data driven product presentation, and performance analysis.&
	
\\ \hline
	\textbf{ Hotkeys:} Establish shortcuts for regular tasks to quickly trigger a Reaction action.&
	
\\ \hline
	\textbf{ Theme Gallery:} Select from a gallery of themes to change the design and experience of your shop.&
	
\\ \hline
	\textbf{ Import from Squarespace:} Import your product catalog from Squarespace into Reaction.&
	
\\ \hline
	 \textbf{ Import from Shopify:} Import your product catalog from Shopify into Reaction.&
	
\\ \hline
	\textbf{ Import from Magento:} Import your product catalog from Magento into Reaction.&
	
\\ \hline
	\textbf{ Import from Spree Commerce:} Import your product catalog from Spree Commerce into Reaction.&
	
\\ \hline
	\textbf{ Import from Big Commerce:} Import your product catalog from Big Commerce into Reaction.&
	
\\ \hline
	&
	
\\ \hline
	&
				
\\ \hline
\end{tabular}
    \caption{ Carta gant}
    \label{tab:task_proyect}
\end{table}

%TODO remover este codigo inecesario
%\section{Estructura de la memoria}\label{cap:intro:estructura}
%La estructura utilizada en este documento para exponer el trabajo realizado es la siguiente:
%
%\begin{itemize}
%	\item \textbf{Capítulo \ref{cap:intro}. \nameref{cap:intro}:} Corresponde a la descripción del tema, la motivación de éste y los alcances y objetivos del trabajo realizado.
%	
%	\item \textbf{Capítulo \ref{cap:antecedentes}. \nameref{cap:antecedentes}:} Corresponde a la revisión bibliográfica o antecedentes. En este capítulo se explican los conceptos necesarios para la comprensión y contextualización del trabajo.
%	
%	\item \textbf{Capítulo \ref{cap:conclusiones}. \nameref{cap:conclusiones}:}	Se enumeran las conclusiones del trabajo realizado y se proponen trabajos a realizar en el futuro.
%\end{itemize}