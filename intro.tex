%%%%%%%%%%%%%%%%%%%%%
%								%
%	ANTECEDENTES GENERALES					%
%								%
%%%%%%%%%%%%%%%%%%%%%

\chapter{Inicio Informe}\label{cap:inicioInforme}

%%%%%%%%%%%%%%%%	MICRORREDES		%%%%%%%%%%%%%%%%%%%%
\section{SQL vs NoSQL: High-Level Differences}\label{cap:antecedentes:datos}


\begin{table}[h!]
    \tiny
   
\begin{tabular}{ |C{0.3\paperwidth}|C{0.3\paperwidth}| }
\hline
SQL & NoSQL \\
\hline
Databases are primarily called as Relational Databases (RDBMS)& 
Database are primarily called as non-relational or distributed database. \\
\hline
Databases are table based databases. Databases represent data in form of tables which consists of n number of rows of data &
databases are document based, key-value pairs, graph databases or wide-column stores. Databases are the collection of key-value pair, documents, graph databases or wide-column stores which do not have standard schema definitions which it needs to adhered to. \\
\hline
Databases have predefined schema &
Databases have dynamic schema for unstructured data.\\
\hline
Databases are vertically scalable. That means are scaled by increasing the horse-power of the hardware. & 
databases are horizontally scalable. That means databases are scaled by increasing the databases servers in the pool of resources to reduce the load.\\
\hline
Databases uses SQL ( structured query language ) for defining and manipulating the data, which is very powerful. &
In NoSQL database, queries are focused on collection of documents. Sometimes it is also called as UnQL (Unstructured Query Language). The syntax of using UnQL varies from database to database. \\ \hline
Examples: MySql, Oracle, Sqlite, Postgres and MS-SQL.&
Examples: MongoDB, BigTable, Redis, RavenDb, Cassandra, Hbase, Neo4j and CouchDb \\
\hline
For complex queries: SQL databases are good fit for the complex query intensive environment &
For complex queries: NoSQL databases are not good fit for complex queries. On a high-level, NoSQL don’t have standard interfaces to perform complex queries, and the queries themselves in NoSQL are not as powerful as SQL query language. \\
\hline
For the type of data to be stored: SQL databases are not best fit for hierarchical data storage. &
NoSQL database fits better for the hierarchical data storage as it follows the key-value pair way of storing data similar to JSON data. NoSQL database are highly preferred for large data set (i.e for big data). Hbase is an example for this purpose. \\
\hline
For scalability: In most typical situations, SQL databases are vertically scalable. You can manage increasing load by increasing the CPU, RAM, SSD, etc, on a single server. &
NoSQL databases are horizontally scalable. You can just add few more servers easily in your NoSQL database infrastructure to handle the large traffic. \\
\hline
For high transactional based application: SQL databases are best fit for heavy duty transactional type applications, as it is more stable and promises the atomicity as well as integrity of the data. &
While you can use NoSQL for transactions purpose, it is still not comparable and sable enough in high load and for complex transactional applications. \\
\hline
For support: Excellent support are available for all SQL database from their vendors. There are also lot of independent consultations who can help you with SQL database for a very large scale deployments. &
For some NoSQL database you still have to rely on community support, and only limited outside experts are available for you to setup and deploy your large scale NoSQL deployments. \\
\hline
For properties: SQL databases emphasizes on ACID properties ( Atomicity, Consistency, Isolation and Durability) &
NoSQL database follows the Brewers CAP theorem ( Consistency, Availability and Partition tolerance ) \\
\hline
For DB types: On a high-level, we can classify SQL databases as either open-source or close-sourced from commercial vendors. &  NoSQL databases can be classified on the basis of way of storing data as graph databases, key-value store databases, document store databases, column store database and \xmlNAME databases. \\
\hline
\end{tabular}
    \caption{SQL vs NoSQL: \textit{High-Level Differences}}
    \label{tab:wide_table}
\end{table}



\section{Bibtex}\label{cap:antecedentes:bibtex}
Bibtex permite usar un archivo bilbio.bib con referencias bibliográficas, de las cuales solo se incluirán el documento las que sean citadas por los comandos:

\cite{Bdo}
\nocite{online}

Donde la primera deja la referencia entre corchetes en el texto, y la segunda no deja referencia en el texto, se utiliza solo para que dicha referencia (online) aparezca en el lista de Biliografía.

En el archivo biblio.bib aparecen algunos ejemplos de tipos de referencias, donde el primer campo es el nombre de la referencia.

Las citas están en español y se ordenan alfabéticamente por autor. Acá otro ejemplo de cita: \cite{Palma}

Ah y ojo que existen subsecciones...

\subsection{Subsección de bibtex}\label{cap:antecedentes:bibtex:subseccion}

\subsubsection{Subsubsección de bibtex}\label{cap:antecedentes:bibtex:subseccion:subsubsection}

