\documentclass[11pt,oneside,letterpaper,leqno]{report}
\usepackage{etex}
\usepackage[utf8]{inputenc}
\usepackage[spanish,es-tabla]{babel}

% Bibliografía en Español
%\usepackage{babelbib}
\usepackage[fixlanguage]{babelbib}
\selectbiblanguage{spanish}

% Fuente Arial
\usepackage[scaled]{uarial}
\renewcommand*\familydefault{\sfdefault}
\usepackage[T1]{fontenc}

% Interlineado 1.5
\usepackage{setspace}
\onehalfspacing

% Márgenes de Página
\usepackage[left=4cm,top=4cm,right=2.5cm,bottom=2.5cm]{geometry}

%%%%%%%%%%%%%%%%%%% region Tablas style %%%%%%%%%%%%%%%%%%%
\usepackage{multirow}
\usepackage{tabularx}
\usepackage{array}
\newcolumntype{L}[1]{>{\raggedright\let\newline\\\arraybackslash\hspace{0pt}}m{#1}}
\newcolumntype{C}[1]{>{\centering\let\newline\\\arraybackslash\hspace{0pt}}m{#1}}
\newcolumntype{R}[1]{>{\raggedleft\let\newline\\\arraybackslash\hspace{0pt}}m{#1}}




%%%%%%%%%%%%%%%%%%% endregion Tablas style %%%%%%%%%%%%%%%%%%%

%%%%%%%%%%%%%%%%%%% region JavaScript style %%%%%%%%%%%%%%%%%%%
\usepackage{listings}
\usepackage{color}
\definecolor{lightgray}{rgb}{.9,.9,.9}
\definecolor{darkgray}{rgb}{.4,.4,.4}
\definecolor{purple}{rgb}{0.65, 0.12, 0.82}

\lstdefinelanguage{JavaScript}{
	keywords={typeof, new, true, false, catch, function, return, null, catch, switch, var, if, in, while, do, else, case, break},
	keywordstyle=\color{blue}\bfseries,
	ndkeywords={class, export, boolean, throw, implements, import, this},
	ndkeywordstyle=\color{darkgray}\bfseries,
	identifierstyle=\color{black},
	sensitive=false,
	comment=[l]{//},
	morecomment=[s]{/*}{*/},
	commentstyle=\color{purple}\ttfamily,
	stringstyle=\color{red}\ttfamily,
	morestring=[b]',
	morestring=[b]"
}

\lstset{
	language=JavaScript,
	backgroundcolor=\color{lightgray},
	extendedchars=true,
	basicstyle=\footnotesize\ttfamily,
	showstringspaces=false,
	showspaces=false,
	numbers=left,
	numberstyle=\footnotesize,
	numbersep=9pt,
	tabsize=2,
	breaklines=true,
	showtabs=false,
	captionpos=b
}

%%%%%%%%%%%%%%%%%%% endregion JavaScript style %%%%%%%%%%%%%%%%%%%

% Glosario
\usepackage{gloss}
\makegloss
\renewcommand\glossname{Glosario}

% Para PS:
%\usepackage{pslatex}
%\usepackage[dvips]{graphicx}

% Para solo PDF:
\usepackage[pdftex]{graphicx}

\DeclareGraphicsRule{.emf}{bmp}{}{}
\DeclareGraphicsExtensions{.pdf,.png,.jpg,.bmp,.gif} %solo para PDFLaTeX

\spanishdecimal{.}

\usepackage{calc}
\usepackage{subfig}
\usepackage{enumerate}
\usepackage{paralist}
\usepackage{url}

%Enumerar secciones de manera particular
\usepackage{sectsty}

%Paquete de dibujo
\usepackage{pgf}
\usepackage{tikz}
\usetikzlibrary{positioning,calc,shapes,arrows,shadows,trees,decorations.pathmorphing,automata}
\usepackage[all]{xy}

%Paquete de dibujo plots 3D
\usepackage{tikz-3dplot}
\newcommand{\dotrule}[1]{\parbox[t]{#1}{\dotfill}}
\usepackage{amsmath}
\usepackage{amssymb}
\usepackage{cite}
\usepackage{multirow}
\usepackage{nameref}
\usepackage[colorlinks=true,linkcolor=black,urlcolor=black,citecolor=black]{hyperref}

\DeclareMathOperator*{\argmax}{arg\,max}

%Paquete para mini-Table-Of-Contents
\usepackage{minitoc}

\usepackage{algorithm}
\usepackage{algorithmic}
\floatname{algorithm}{Algoritmo}

%para código fuente
\newenvironment{codigoenv}
{\fontsize{10pt}{12pt} \linespread{1}} { \normalsize}

%INDENTACIÓN
\parindent=8mm

%ESPACIADO PARRAFOS
\setlength{\parskip}{3mm}
\setlength{\footskip}{10mm}
\setlength{\headsep}{10mm}

%Titulo para tabla de contenido pequeñas
\def\mtctitle{Contenido}

% Evitar separación de palabras
\pretolerance=10000
\tolerance=10000

%%%%%%%%%%%%%%%%%%% region LayersDiagram style %%%%%%%%%%%%%%%%%%%

\definecolor{mybluei}{rgb}{.48,.6,.8}
\definecolor{myblueii}{rgb}{.28,.47,.75}
\definecolor{mygreen}{rgb}{.78,.85,.5}
\definecolor{mywhite}{rgb}{1,1,1}


\pgfdeclarelayer{background}
\pgfsetlayers{background,main}
            
%%%%%%%%%%%%%%%%%%% endregion LayersDiagram style %%%%%%%%%%%%%%%%%%%

%%%%%%%%%%%%%%%%%%% region Fix space \newcommand bug %%%%%%%%%%%%%%%%%%%
\usepackage{xspace}
%%%%%%%%%%%%%%%%%%% endregion Fix space \newcommand bug %%%%%%%%%%%%%%%%%%%